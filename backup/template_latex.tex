%%%%%%%%%%%%%%%%%%%%%%%%%%%%%%%%%%%%%%%%%
% LaTeX Template
% Versão 1.0 (29/05/15)
%  
%%%%%%%%%%%%%%%%%%%%%%%%%%%%%%%%%%%%%%%%%

\documentclass[pdftex,12pt,a4paper]{article} % Esta deve ser SEMPRE a primeira linha
% válida no documento.
% Tudo que estiver na linha depois do símbolo % é comentário
%----------------------------------------------------------------------------------------
%	Configurações e pacotes utilizados no manuscrito
%----------------------------------------------------------------------------------------
% Este são os pacotes utilizados no documento. Dependendo da necessidade, você pode
% acrescentar or remove-los
\usepackage[pdftex]{graphicx}
\usepackage[T1]{fontenc}
\usepackage[utf8]{inputenc}
\usepackage[brazil]{babel}
%\usepackage{pslatex}
%\usepackage{pdfpages}
\usepackage{achicago}
\usepackage{framed}
\usepackage{color}

\renewcommand{\baselinestretch}{1.2}
%\paperwidth 8.26in % A4 (210 mm)
%\paperheight 11.69in % A4 (297 mm)
%\textwidth 6.26in
%\textheight 9.7in
%\columnsep 0.0in
%\oddsidemargin .0in
%\topmargin -.2in
%\headheight 0in
%\headsep 0in
%\topskip 0in
 
% for fancyboxes
\setlength{\fboxrule}{0.2mm}
\setlength{\fboxsep}{2ex}

%----------------------------------------------------------------------------------------
% Coloca a tabela ou a figura no local onde você quer.
\makeatletter
\newenvironment{tablehere}
  {\def\@captype{table}} {}
 
\newenvironment{figurehere}
  {\def\@captype{figure}}
  {}
\makeatother 

%----------------------------------------------------------------------------------------
% Em caso do latex não conseguir fazer a separação silábica corretamente, defina aqui:
\hyphenation{des-cri-tos}
\hyphenation{in-ter-na-cio-nal}

\begin{document}
\pagestyle{empty}
%----------------------------------------------------------------------------------------
%	Capa  ou página com o título.
%----------------------------------------------------------------------------------------

\title{\Large Modelo básico para produzir um artigo em Latex}
\author{\large Olga Sato\thanks{olga.sato@usp.br} \hspace{.25cm} \\
Departamento de Oceanografia Física, Química e Geológica \\
Instituto Oceanogr\'afico - Universidade De S\~ao Paulo\\
São Paulo, SP}
\date{\today}
\thispagestyle{empty}
\maketitle
\normalsize

%----------------------------------------------------------------------------------------
% Comando para mudar de página.
%
\newpage

%----------------------------------------------------------------------------------------
% Criar uma página com o sumário. Se não quiser, é só comentar as linhas.
%\thispagestyle{empty}
%\tableofcontents
%\thispagestyle{empty}
%\pagestyle{plain}

%----------------------------------------------------------------------------------------
%	ABSTRACT PAGE
%----------------------------------------------------------------------------------------
\begin{center}
{\Large Modelo básico para produzir um artigo em Latex}
\end{center}

\vspace{-.25in}
\noindent\hrulefill

\vspace{.5in}

\centerline{\large Olga T. Sato}

\begin{abstract}

  Este é um curso básico sobre como utilizar o {\bf latex} para
  produzir um documento de altíssima qualidade tipográfica.  O latex é
  um programa que processa uma série de comandos e definições
  colocados num texto e transforma--os num documento com qualidade
  compatível com aquela encontrada em livros e artigos científicos
  publicados.

  O intuito deste curso não é fornecer um guia completo com todos as
  capacidades que o latex proporciona. Isso é impossível. Para isso,
  verifique na internet a quantidade de informações sobre latex que
  está disponível. Neste curso, iremos cobrir alguns elementos básicos
  de como editar um artigo, como criar seções de um artigo, fazer
  listas, tabelas, incluir figuras, escrever equações e
  principalmente, como incluir referências bibliográficas. 
  
  O roteiro para este curso será o seguinte:
  \begin{enumerate}
    \item Entender o funcionamento do latex examinando o presente
      documento. Salvem este documento com um outro nome e acrescentem
      seus próprios comentários para lembrar mais tarde.
    \item Transformar um documento que vocês trouxeram (pode ser .doc)
      num documento PDF feito pelo latex.
  \end{enumerate}
  
\begin{framed}
  Para gerar um arquivo .pdf, siga o seguinte roteiro:
  \begin{enumerate}
    \item Edite um arquivo com extensão .tex. Por exemplo, salve este
      documento como \linebreak{nomeartigo.tex}, sem formatação.
      \item Para executar o latex, use: \\
      \verb#pdflatex nomeartigo#
  \end{enumerate}
\end{framed}
  
\end{abstract}


\clearpage
\newpage
%----------------------------------------------------------------------------------------
%	Sumário, lista de figuras e tabelas
%----------------------------------------------------------------------------------------

\pagestyle{empty}

\tableofcontents % O latex faz um sumário com todas as suas seções

\listoffigures % lista das figuras

\listoftables % lista das tabelas

\clearpage
\newpage
%----------------------------------------------------------------------------------------
%	O conteudo propriamente dito
%----------------------------------------------------------------------------------------
\section{Primeira Seção: a forma do texto}
\label{psecao}

Sed ut perspiciatis unde omnis iste natus error sit voluptatem
accusantium {\bf negrito se faz assim} doloremque laudantium, totam
rem aperiam, eaque ipsa quae ab illo inventore veritatis et quasi
architecto beatae vitae dicta sunt explicabo. Nemo enim ipsam
voluptatem quia voluptas sit aspernatur aut odit aut fugit, sed quia
consequuntur magni dolores eos qui ratione voluptatem sequi
nesciunt. Neque porro quisquam est, qui dolorem ipsum quia dolor sit
amet, consectetur, {\em também dá para fazer itálico} adipisci velit,
sed quia non numquam eius modi tempora incidunt ut labore et dolore
magnam aliquam quaerat voluptatem. Ut enim ad minima veniam, quis
nostrum exercitationem ullam corporis suscipit laboriosam, nisi ut
aliquid ex ea commodi consequatur? Quis autem vel eum iure
reprehenderit qui in ea voluptate velit esse quam nihil molestiae
consequatur, vel illum qui dolorem eum fugiat quo voluptas nulla
pariatur? {\sc Tudo em letra maiúscula}. At vero eos et accusamus et
iusto odio dignissimos ducimus qui blanditiis praesentium voluptatum
deleniti atque corrupti quos dolores et quas molestias excepturi sint
occaecati cupiditate non provident, similique sunt in culpa qui
officia deserunt mollitia animi, id est laborum et dolorum fuga. Et
harum quidem rerum facilis est et expedita distinctio. Nam libero
tempore, cum soluta nobis est eligendi optio cumque nihil impedit quo
minus id quod maxime placeat facere

\subsection{Subseção}

{\Large Letra grande} Lorem ipsum dolor sit amet, consectetuer
adipiscing elit. {\large Letra um pouco menor}. Aenean commodo ligula
eget dolor. Aenean massa. Cum sociis natoque penatibus et magnis dis
parturient montes, nascetur ridiculus mus. {\LARGE Bem grande!} Donec
quam felis, ultricies nec, pellentesque eu, pretium quis, sem. Nulla
consequat massa quis enim. {\small Letra pequena.} Donec pede justo,
fringilla vel, aliquet nec, vulputate eget, arcu. In enim justo,
rhoncus ut, imperdiet a, venenatis vitae, justo. {\tiny Consegue ler?}
Nullam dictum felis eu pede mollis pretium. Integer tincidunt. Cras
dapibus.

\subsection{Nova seção}

Vivamus elementum semper nisi. Aenean vulputate eleifend
tellus. {\color{blue}Dá para trocar a cor do texto.} Aenean leo
ligula, porttitor eu, consequat vitae, eleifend ac, enim. Aliquam
lorem ante, dapibus in, viverra quis, feugiat a, tellus. Phasellus
viverra nulla ut metus varius laoreet. Quisque rutrum. Aenean
imperdiet. Etiam ultricies nisi vel augue. Curabitur ullamcorper
ultricies nisi. {\color{red} Agora em vermelho.} Nam eget dui. Etiam
rhoncus. Maecenas tempus, tellus eget condimentum rhoncus, sem quam
semper libero, sit amet adipiscing sem neque sed ipsum. Nam quam nunc,
blandit vel, luctus pulvinar, hendrerit id, lorem. Maecenas nec odio
et ante tincidunt tempus. Donec vitae sapien ut libero venenatis
faucibus. Nullam quis ante. Etiam sit amet orci eget eros faucibus
tincidunt. Duis leo. Sed fringilla mauris sit amet nibh. Donec sodales
sagittis magna. Sed consequat, leo eget bibendum sodales, augue velit
cursus nunc,

\subsubsection{Subsubseção}

Lorem ipsum dolor sit amet, consectetuer adipiscing elit. Aenean
commodo ligula eget dolor. Aenean massa. Cum sociis natoque penatibus
et magnis dis parturient montes, nascetur ridiculus mus. Donec quam
felis, ultricies nec, pellentesque eu, pretium quis, sem. Nulla
consequat massa quis enim. Donec pede justo, fringilla vel, aliquet
nec, vulputate eget, arcu. In enim justo, rhoncus ut, imperdiet a,
venenatis vitae, justo. Nullam dictum felis eu pede mollis
pretium. Integer tincidunt. Cras dapibus. Vivamus elementum semper
nisi.

{\bf Para criar um parágrafo novo, é só deixar uma linha em branco.}
Aenean vulputate eleifend tellus. Aenean leo ligula, porttitor eu,
consequat vitae, eleifend ac, enim. Aliquam lorem ante, dapibus in,
viverra quis, feugiat a, tellus. Phasellus viverra nulla ut metus
varius laoreet. Quisque rutrum. Aenean imperdiet. Etiam ultricies nisi
vel augue. Curabitur ullamcorper ultricies nisi. Nam eget dui. Etiam
rhoncus. Maecenas tempus, tellus eget condimentum rhoncus, sem quam
semper libero, sit amet adipiscing sem neque sed ipsum. Nam quam nunc,
blandit vel, luctus pulvinar, hendrerit id, lorem. Maecenas nec odio
et ante tincidunt tempus. Donec vitae sapien ut libero venenatis
faucibus. Nullam quis ante. Etiam sit amet orci eget eros faucibus
tincidunt. Duis leo. Sed fringilla mauris sit amet nibh. Donec sodales
sagittis magna. Sed consequat, leo eget bibendum sodales, augue velit
cursus nunc,


\paragraph{Parágrafo}

Lorem ipsum dolor sit amet, consectetuer adipiscing elit. Aenean
commodo ligula eget dolor. Aenean massa. Cum sociis natoque penatibus
et magnis dis parturient montes, nascetur ridiculus mus. Donec quam
felis, ultricies nec, pellentesque eu, pretium quis, sem. Nulla
consequat massa quis enim. Donec pede justo, fringilla vel, aliquet
nec, vulputate eget, arcu. In enim justo, rhoncus ut, imperdiet a,
venenatis vitae, justo. Nullam dictum felis eu pede mollis
pretium. Integer tincidunt. Cras dapibus. Vivamus elementum semper
nisi. Aenean vulputate eleifend tellus. Aenean leo ligula, porttitor
eu, consequat vitae, eleifend ac, enim.

{\bf Esse parágrafo é um elemento dentro da divisão do
  artigo. Utilize--o quando quiser criar uma subdivisão abaixo da
  subsubseção.}

\section{Segunda seção: Um controle maior do texto}

Veja como se faz para colocar uma referência a alguma seção, figura,
tabela, etc. que está definida no texto. Fazendo assim, você pode
direcionar o leitor para alguma informação em específico, por exemplo,
como mudar o tamanho da fonte, {\bf conforme visto na
  seção~\ref{psecao}}.

\subsection{Fazendo listas}

Você vai precisar enumerar, listar e descrever itens ao longo do
texto. Veja como se faz.

Fazer lista como itens sem numeração:
\begin{itemize}
  \item Vivamus elementum semper nisi. Aenean vulputate eleifend
tellus. Aenean leo ligula, porttitor eu, consequat vitae, eleifend ac,
enim.
\item Aliquam lorem ante, dapibus in, viverra quis, feugiat a,
tellus. Phasellus viverra nulla ut metus varius laoreet. Quisque
rutrum.
\item Aenean imperdiet. Etiam ultricies nisi vel augue. Curabitur
ullamcorper ultricies nisi. Nam eget dui. Etiam rhoncus. Maecenas
tempus, tellus eget condimentum rhoncus, sem quam semper libero, sit
amet adipiscing sem neque sed ipsum.
\end{itemize}

A mesma lista, só que você escreve algo no lugar da bolinha:
\begin{itemize}
  \item[\bf Centauri:] Vivamus elementum semper nisi. Aenean vulputate eleifend
tellus. Aenean leo ligula, porttitor eu, consequat vitae, eleifend ac,
enim.
\item[\bf Minbari:] Aliquam lorem ante, dapibus in, viverra quis, feugiat a,
tellus. Phasellus viverra nulla ut metus varius laoreet. Quisque
rutrum.
\item[\bf Vorlon:] Aenean imperdiet. Etiam ultricies nisi vel augue. Curabitur
ullamcorper ultricies nisi. Nam eget dui. Etiam rhoncus. Maecenas
tempus, tellus eget condimentum rhoncus, sem quam semper libero, sit
amet adipiscing sem neque sed ipsum.
\end{itemize}
Fazer lista numerada: 
\begin{enumerate}
 \item Nam quam nunc, blandit vel,
luctus pulvinar, hendrerit id, lorem. Maecenas nec odio et ante
tincidunt tempus.
\item Donec vitae sapien ut libero venenatis
faucibus. Nullam quis ante. Etiam sit amet orci eget eros faucibus
tincidunt.
\item Duis leo. Sed fringilla mauris sit amet nibh. Donec sodales
sagittis magna. Sed consequat, leo eget bibendum sodales, augue velit
cursus nunc,
\end{enumerate}

Só descrevendo...
\begin{description}
   \item Nam quam nunc, blandit vel,
luctus pulvinar, hendrerit id, lorem. Maecenas nec odio et ante
tincidunt tempus.
\item Donec vitae sapien ut libero venenatis
faucibus. Nullam quis ante. Etiam sit amet orci eget eros faucibus
tincidunt.
\item Duis leo. Sed fringilla mauris sit amet nibh. Donec sodales
sagittis magna. Sed consequat, leo eget bibendum sodales, augue velit
cursus nunc,
\end{description}


%----------------------------------------------------------------------------------------
\subsection{Hora de aprender a escrever equações}

Vamos começar com algo simples. Vamos escrever a segunda lei de Newton:
\begin{equation}
  F = ma
  \label{newton1}
\end{equation}
só que tinha que escrever com vetor:
\begin{equation}
  \vec{F} = m \vec{a} 
  \label{newton2}
\end{equation}

Agora tentem escrever a equação da figura abaixo:

\centerline{\hbox{\includegraphics[width=.4\textwidth]{ep.jpg}}}

Para ajudar, já comecei a escrever a equação. Agora é só completar.
\begin{equation}
  E_p=
\label{ep}
\end{equation}
onde $\rho$ é a densidade da fluido, $g$ é a aceleração da gravidade,
$a$ é a amplitude, $k$ o número de onda, $\omega$ é a frequência.

{\bf Nota: Uma equação é como se fosse uma frase. Então, ela tem que
  ter pontuação.}

Quando quiser escrever equações que precisam ser alinhadas de alguma
forma, utilize o \verb#eqnarray#:
\begin{eqnarray}
\frac{\partial \eta}{\partial t} + H \frac{\partial u}{\partial x}
& = & 0 \nonumber \\ 
\frac{\partial u}{\partial t} & = & -g \frac{\partial
  \eta}{\partial x},   
\label{swater_kelvin} \\
fu & = & -g \frac{\partial \eta}{\partial y}  \nonumber 
\end{eqnarray}

\subsection{Incluindo tabelas}


A Tabela~\ref{crono} mostra um exemplo de execução do projeto que
deverá ter a duração de 24 meses discriminado por trimestre:
\begin{table}[ht]
  \begin{center}
\caption{Cronograma de execução por trimestre.}
\begin{tabular}{ccccccc} \hline \hline
 Abr & Mai & III & IV & V & VI & VII \\ \hline
 1  & 2, 3 &  3, 4,  &   4     &   5 &  5, 6 & 7 \\ \hline
\end{tabular}
\label{crono}
  \end{center}
\end{table}

Algumas coisas importantes a se notar na hora de fazer tabela:
\begin{itemize}
\item Normalmente, a legenda da tabela vem em cima; a de uma figura vem abaixo.
  \item Observe como é colocada uma identificação na tabela usando--se
    o \verb#\label#. E note que no texto, quando a tabela é chamada por esse
    nome, aparece um número.
\item Você não precisa se preocupar em colocar a tabela no lugar
  certo. O latex processa o texto e coloca onde julgar ser o melhor
  lugar.
\item Caso você queira ter um melhor controle, ao invés de table, use
  o tablehere.
\begin{tablehere}
\begin{center}
\caption{Wind data period range.}
\begin{tabular}{|l|c|} \hline \hline
 Source &  Period \\ \hline
ERS & 01/1992 -- 01/2001 \\
SSM/I & 01/1988 -- 12/2001 \\
NCEP & 01/1988 -- 09/2003 \\
Quikscat & 09/1999 -- 08/2003 \\ \hline \hline
\end{tabular}
\end{center}
\label{period_range}
\end{tablehere}
\end{itemize}

\subsection{Incluindo figuras}

\begin{figure}
\vspace{.5cm}
\centerline{\hbox{\includegraphics[width=.6\textwidth]{perigaud_nino3.jpg}}}
\vspace{-.5cm}
\caption{\small Índice Niño 3.}
\label{elnino}
\end{figure}

Algumas observações importantes sobre figuras:
\begin{itemize}
\item Legenda vem depois da figura.
  \item Você pode controlar o tamanho da figura no próprio
    latex. Observe que você pode mudar o tamanho da figura usando o
    \verb#[width=\textwidth]# ou \verb#[length=\textlength]#. Não é
    bom usar os dois ao mesmo tempo pois senão mudará a razão de
    aspecto da figura. Nesse argumento, coloque o tamanho desejável,
    em centímetros, polegadas, pontos, ou usar textwidth or textheight
    (pode colocar um valor entre 0 e 1 para escalar a figura).
  \item Os espaços acima e abaixo da figura podem ser mudados com o
  comando \verb#\vspace#. Use esse recurso somente quando realmente
  necessário. É desejável que as figuras já tenham uma margem pequena
  assim você não precisa ficar ajustando uma a uma.
\item Note que o número da figura é colocada automaticamente. Podemos
  nos referir a essa figura com Figura~\ref{elnino}.
\end{itemize}

Podemos colocar duas figuras juntas com uma legenda somente:
\begin{figure}[ht]
\vspace{.5cm}
\centerline{\hbox{\includegraphics[width=.5\textwidth]{perigaud_nino3.jpg}}
{\includegraphics[width=.5\textwidth]{perigaud_nino3.jpg}}}
\centerline{\hbox{\includegraphics[width=.5\textwidth]{perigaud_nino3.jpg}}
{\includegraphics[width=.5\textwidth]{perigaud_nino3.jpg}}}
\vspace{-.5cm}
\caption{\small Índice Niño 3 duas vezes.}
\label{elnino2}
\end{figure}

\section{Como incluir referências bibliográficas}

O latex possui um recurso incrível para se incluir referências
bibliográficas. Ele se baseia num sistema de ``fichas'' como se tinha
em bibliotecas antigamente.  Em cada ``ficha'' são colocados os
detalhes de um artigo. Vejam o exemplo do reference.bib que estou
compartilhando com vocês. Você pode ter um arquivo reference.bib que
poderá ser usado para sempre. Adicione quantas ``fichas''
quiser. Junte com a de seus amigos.

Uma vez que a ``ficha'' do artigo estiver inclusa no reference.bib,
essa ficha pode ser citada no artigo. O latex criará um arquivo
especial com todos as referências, colocando a chamada no texto e no
final do artigo, na lista de referências.

Para exemplificar, vamos nos referir ao artigo de \shortciteN{Sato00}
que trata sobre fluxo meridional de calor no Atlântico Norte. O
formato da citação pode mudar para cada caso específico:
\begin{itemize}
  \item \shortcite{Sato00}
  \item \shortciteN{Sato00}
  \item \shortciteNP{Sato00}
\end{itemize}

\begin{framed}
Receita para incluir as referências no artigo:
\begin{enumerate}
\item Tenha o reference.bib pronto com cada uma das citações que vai utilizar;
  \item Escolha o estilo de referência que usará. No começo do seu
    arquivo .tex, inclua o pacote com: \verb#\usepackage{achicago}#;
  \item Antes do final do documento .tex, chame o arquivo das
    referências e o estilo:\\
    \verb#\bibliography{references}#\\
    \verb#\bibliographystyle{achicago}#
\item Inclua no seu artigo todas as citações desejadas. 
\item Execute o latex: \verb#pdflatex nomeartigo#.\label{latexe}
  \item Execute o bibtex: \verb#bibtex nomeartigo#. Cada vez que
    fizer inclusões de referências, você deve executar o bibtex para
    atualizar um arquivo que será criado no mesmo diretório com a
    extensão .bbl. Por curiosidade, verifique esse arquivo.
\item Execute o item~\ref{latexe} desta lista umas duas vezes para que
  latex consiga atualizar as suas referências internas.
\end{enumerate}
\end{framed}

%----------------------------------------------------------------------------------------
%	BIBLIOGRAFIA
%----------------------------------------------------------------------------------------

\bibliography{references}
\bibliographystyle{achicago}

\end{document}
