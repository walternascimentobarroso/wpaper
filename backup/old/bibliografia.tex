\section{Bibliografia}
A forma mais comun e simples de identificar uma tabela, não so no postgresql, mas como em todos os SGBDs, é com o uso de ID.

Usar ID, é bastante util e simples, mas quando a aplicação passar de algo simples, para algo bem profissional e robusto, 
é necessario tomar um certo cuidado.

O principal problema por usar ID como identificador primário, é a falhar de segurança, onde todos os identifcadores são sequenciais,
logo ao ter conhecimento de um, é possivel por forçar bruto encontrar os outros.

Existem diversas formas de não sofrer um ataque baseado na falha de um ID sequencial, e uma delas é usando o UUID.