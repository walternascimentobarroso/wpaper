\documentclass{article}

\usepackage[portuges]{babel}
\usepackage[latin1]{inputenc}

\parindent=0pt
\parskip=2pt

\title{Um Exemplo de Artigo em \LaTeX}
\author{Sandra Lopes\thanks{Bolseiro da FCT}\\Escola FFF\\ Fafe 
        \and 
        Pedro Henriques\\Escola de Engenharia\\ Braga }
\date{Outubro de 2002}

\begin{document}

\maketitle

\begin{abstract}
\noindent Isto � um resumo do artigo.\\
O objectivo � sintetisar em 2 ou 3 par�grafos a ideia principal descrita no artigo.
\end{abstract}


\section{Introdu��o}

\section{Background}
\subsection{Conceitos b�sicos em XXX}
\subsubsection{Conceitos b�sicos em XXX.1}
Aqui vai um exemplo de uma lista numerada
\begin{enumerate}
\item primeira caracter�stica desta fase
\item segunda caracter�stica desta fase
\item terceira caracter�stica desta fase
\end{enumerate}

\subsubsection{Conceitos b�sicos em XXX.2}
Seguem-se algumas defini��es fundamentais para se perceberem as ideias 
defendidas a seguir:
\begin{description}
\item[conceito 1] respectiva defini��o 1
\item[conceito 2] descri��o do  conceito 2
\end{description}

\subsection{Conceitos b�sicos em YYY}

\section{A Proposta}

\section{A sua Implementa��o}

\section{Conclus�o}
S�ntese do que foi dito.\\
Lista dos resultados atingidos:
\begin{itemize}
\item resultado 1
\item resultado 2
\end{itemize}
Conclus�o final e Trabalho Futuro.
\end{document}
