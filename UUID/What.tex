\section{O que é UUID?}
UUID é a sigla em ingês para Universally unique identifier(Identificador Único Universal).
O UUID foi padronizado pela Open Software Foundation como um identificador padrão para softwares.
O objetivo é criar um idenficador único que possa ser compartilhado com outros softwares, 
facilitando a troca de informações entre sistemas. 

A definição para um UUID é um número de 128 bits.
Em teoria, o número possível de UUIDs geradas é de $3 \times 10^{38}$.

Atualmente, a maioria das aplicações usam UUID para identificar, pois com isso tem um registro unico
em diversos lugares, um bom exemplo para este caso é o uso de UUID para identificar album de música.