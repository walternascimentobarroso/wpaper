\documentclass[a4paper]{article}
\usepackage[utf8]{inputenc} % no Windows pode usar [latin1], mas o TeXWorks aceita utf8.
\usepackage[T1]{fontenc}
\usepackage[brazil]{babel}
\usepackage{hyperref}

% Define comando BibTeX
\def\BibTeX{{\rm B\kern-.05em{\sc i\kern-.025em b}\kern-.08em
    T\kern-.1667em\lower.7ex\hbox{E}\kern-.125emX}}

\title{Bibliografia com \BibTeX}
\author{http://latexbr.blogspot.com.br/}
\date{\the\year} % ou \today

\begin{document}
\maketitle



\subsection{Adicionando itens a bibliografia}

Pesquise as refer\^encias em \href{http://scholar.google.com.br/schhp?hl=pt-BR}{Google Acad\^emicos}. Clique em Configura\c c\~oes (lado superior direito da tela) e marque 'Mostre links para importar cita\c c\~oes para o \BibTeX'. A partir dai pesquise a refer\^encia e clique em 'Importe para o \BibTeX'.

\'E necess\'ario que a refer\^encia seja citada no documento, para isso digite \verb|\cite{}|. \cite{oetiker}, \cite{lamport}, \cite{mittelbach}, \cite{tantau}
Veja em \textit{refs.bib} algumas refer\^encias.

Para compilar com refer\^encia abra o terminal e digite:

\begin{verbatim}
pdflatex exemplo
bibtex exemplo
pdflatex exemplo
pdflatex exemplo
\end{verbatim}

Você também pode compilar usando\href{http://latexbr.blogspot.com.br/search/label/Compilador}{LaTeXmk ou Rubber}. 

Veja também \href{http://latexbr.blogspot.com.br/2012/07/cartao-com-principais-comandos-do-latex.html}{latexrefcard}.

\href{http://latexbr.blogspot.com.br/}{LaTeXBR}

\bibliographystyle{abbrv}
\bibliography{refs}

\end{document}