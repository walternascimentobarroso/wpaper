\documentclass[mscexam,pdftex]{ppgi-ufam}
\usepackage[brazil]{babel}
\usepackage[utf8]{inputenc}
\usepackage[T1]{fontenc}
\let\newfloat=\undefined
\usepackage{longtable}
\usepackage{caption}
\usepackage{listings}
\usepackage{pslatex}
\usepackage{setspace}
\usepackage[all]{hypcap}
\usepackage[nottoc]{tocbibind}
\usepackage[square,sort,nonamebreak,comma]{natbib}
\usepackage{epstopdf}
\usepackage{scalefnt}
\usepackage{multirow}
\newcommand{\tab}{\hspace*{0.5em}}
\usepackage{float}
\usepackage{tabularx}
\usepackage{comment}
\usepackage{booktabs}
\usepackage{lscape}
\addto{\captionsbrazil}{
    \renewcommand{\bibname}{Refer\^{e}ncias}
}

\newcommand{\minitab}[2][l]{\begin{tabular}{#1}#2\end{tabular}}

\makelosymbols
\makeloabbreviations
\hypersetup{hidelinks}

\begin{document}

\title{Tutor inteligente com raciocínio baseado em casos para deficientes visuais na plataforma moodle}
\foreigntitle{Intelligent tutor with case-based reasoning for the visually impaired on the moodle platform}
\author{Walter Nascimento}{Barroso}
\advisor{Profa.}{Elaine Harada Teixeira de}{Oliveira}{D.Sc.}{Instituto de Computação}{Universidade Federal do Amazonas}
\examiner{Prof.}{xxx}{D.Sc.}{Instituto de Computação}{Universidade Federal do Amazonas}
\examiner{Profa.}{xxx}{D.Sc.}{Instituto de Computação}{Universidade Federal do Amazonas}
\examiner{Prof.}{xxx}{D.Sc.}{Instituto de Computação}{Universidade Federal do Amazonas}
\examiner{Prof.}{xxx}{D.Sc.}{Instituto de Computação}{Universidade Federal do Amazonas}
\department{IComp}
\date{03}{2017}

\keyword{Raciocínio baseado em casos}
\keyword{Educação}
\keyword{E-learning}
\keyword{Educação a distância}
\keyword{Deficientes visuais}
\keyword{Ensino online}
\keyword{Inteligência artificial}

\maketitle

\dedication{A Deus, \\ aos professores, \\ aos colegas de curso e \\ aos meus familiares.}

\chapter*{Agradecimentos}
Agradeço a Deus.

\begin{abstract}
Esta pesquisa investiga ... .
\linebreak
\linebreak
\textbf{Palavras-chave:} Raciocínio baseado em casos, Educação, E-learning, Educação a distância, Deficientes visuais, Ensino online, Inteligência artificial.
\end{abstract}

\begin{foreignabstract}
This research investigates ... .
\linebreak
\linebreak
\textbf{Keywords:} Case-based Reasoning, Education, E-learning, Distance Education, Visually impaired, Online Teaching, Artificial Intelligence.
\end{foreignabstract}

\setcounter{tocdepth}{1}

\tableofcontents
\listoffigures
\listoftables

\mainmatter

\section{Introdução}
A forma mais comun e simples de identificar uma tabela, não so no postgresql, mas como em todos os SGBDs, é com o uso de ID.

Usar ID, é bastante util e simples, mas quando a aplicação passar de algo simples, para algo bem profissional e robusto, 
é necessario tomar um certo cuidado.

O principal problema por usar ID como identificador primário, é a falhar de segurança, onde todos os identifcadores são sequenciais,
logo ao ter conhecimento de um, é possivel por forçar bruto encontrar os outros.

Existem diversas formas de não sofrer um ataque baseado na falha de um ID sequencial, e uma delas é usando o UUID.
\chapter{Fundamentação Teórica} \label{chap:Fundamentação}
\lettrine{N}{este} capítulo são apresentados os conceitos fundamentais levantados através de uma revisão sistemática da literatura (RSL) e complementado por uma pesquisa exploratória. 

Desta forma, são apresentados os conceitos sobre raciocínio baseado em casos, ambiente virtual de aprendizagem e educação inclusiva de deficientes visuais.

\section{Raciocínio Baseado em Casos}
\lettrine{A} idéia principal do RBC é resolver cada novo problema lembrando das soluções de situações anteriores similares.

A definição clássica de um sistema RBC foi elaborada por Reisbeck e Schank (1989): “Um
sistema RBC resolve problemas, adaptando soluções que foram utilizadas para resolver
problemas anteriores”.

Um exemplo seria, uma pessoa escolhe comprar um objeto, e com base nas experiências
passadas decide qual será o melhor para não passar por problemas já vividos com outros
objetos.

Wangenheim et al (2013) afirmam que dá-se o nome de Raciocínio Baseado em Casos (RBC), à técnica de Inteligência Artificial, como um conjunto de atividades que auxilia na resolução de problemas, propondo soluções incontestavelmente utilizadas e documentadas, ao recuperar e adaptar experiências passadas – chamadas casos – armazenadas em uma Base de Casos. Um novo caso é resolvido com base na adaptação de solução de casos similares já conhecidos (Wangenheim et al 2013).

O ciclo de funcionamento de um sistema de RBC é composto por quatro etapas de execução, conhecida como 4R, conforme definido por Aamodt `I\&'  Plaza (1994), explicado abaixo e ilustrado pela Figura 1.
\begin{itemize}
    \item Recuperação: a partir da apresentação ao sistema de um novo problema é feita a recuperação na base de casos daquele mais parecido com o problema em questão. Isto é feito a partir da identificação das características mais significantes em comum entre os casos;
    \item Reuso: a partir do caso recuperado é feita a reutilização da solução associada àquele caso. Geralmente a solução do caso recuperado é transferida ao novo problema diretamente como sua solução;
    \item Revisão: é feita quando a solução não pode ser aplicada diretamente ao novo problema. O sistema avalia as diferenças entre os problemas (o novo e o recuperado), quais as partes do caso recuperado são semelhantes ao novo caso e podem ser transferidas adaptando assim a solução do caso recuperado da base à solução do novo caso;
    \item Retenção: é o processo de armazenar o novo caso e sua respectiva solução para futuras recuperações. O sistema irá decidir qual informação armazenar e de que forma.
\end{itemize}

\begin{figure}[htbp!]
\centering
\includegraphics[scale=0.6]{img/cicloRBC.png}
\caption{Ciclo do Raciocínio Baseado em Casos.}
\label{fig:cicloRBC}
\end{figure}

Conforme Wangenheim (2003), as etapas mais importantes do processo de
desenvolvimento de um sistema RBC são:
\begin{itemize}
    \item Aquisição de Conhecimento;
    \item Representação de Caso;
    \item Indexação;
    \item Recuperação de Casos;
    \item Adaptação de Casos.
\end{itemize}

\subsection{A Aquisição de Conhecimento}
\lettrine{A}{quisição} de conhecimento é uma etapa muito importante no desenvolvimento do sistema de RBC. Conforme Kolodner (1993) ela é considerada o componente crítico no desenvolvimento de sistemas de RBC. Consiste na seleção de casos que irão formar uma base de informações (um sistema de banco de dados – uma base de casos) que contenha implicitamente o conhecimento necessário na solução de problemas;

\subsection{Representação de caso}
\lettrine{U}{m} caso é definido pela representação do conhecimento contido em uma experiência vivida que dirige o indivíduo a alcançar seus objetivos (Leake, 1996). Todo caso é composto por: Problema: descreve o estado do mundo real onde o caso ocorre; e, Solução: contém o estado das soluções derivadas para o problema (Watson, 1997).Pode-se armazenar os casos utilizando os diferentes tipos usados pelos computadores, arquivos, banco de dados, etc.

\subsection{Indexação}
\lettrine{I}{ndexação} é um processo importante e decisivo dentro do RBC, assim como numa base de dados convencional, porém não necessariamente, todas as informações relevantes deverão ser indexadas. As informações indexadas servirão para acelerar o processo de recuperação enquanto que informações não-indexadas podem servir de contextualização para a decisão de reutilização do caso ou em outros aspectos do sistema de informação e não somente na recuperação. [VITORINO 2009] Uma das tarefas da indexação é atribuir pesos às características dos casos, para que possa alcançar a recuperação de casos utilizando o algoritmo de vizinhança (Nearest Neighbor, Michie 1994). Com esta técnica define peso as características mais importantes.

\subsection{Recuperação de Casos}
\lettrine{O} processo de recuperação de casos (Riesbeck, 1999) inicia com uma descrição de problema e finaliza quando um melhor caso for encontrado. O sistema procura na base de casos, o caso mais similar com o novo problema.

Leake (1996) coloca que uma característica importante dos sistemas de RBC é possuir alternativas para identificar os casos a fim de conseguir representá-los e indexá-los, garantindo que sejam recuperados os mais úteis para resolver o problema do usuário. Somente consegue-se alternativas para identificar os casos através de procedimentos de comparação e medição de similaridades.

\section{Ambiente Virtual de Aprendizagem}
\lettrine{O}{s} AVA são espaços virtuais que permitem aos sujeitos envolvidos nos processos de ensino, aprendizagem e avaliação a busca por conhecimentos e capacitação (MACIEL, 2012). São projetados para facilitar o acesso a materiais de aprendizagem e a comunicação dos estudantes entre si e com os professores. Em resumo, refere-se a um espaço no qual
ocorre a comunicação pedagógica em processos formativos semipresenciais ou a distância (ADELL; BELLVER; BELLVER, 2010). Segundo Almeida (2003), os ambientes virtuaispermitem integrar múltiplas mídias, linguagens e recursos, apresentar informações de maneira organizada e desenvolver interações entre pessoas. Oferecem possibilidades para
a criação de espaços educacionais diferenciados que valorizam a participação do aluno de forma mais contextualizada e integrada aos objetivos de aprendizagem (KENSKY, 2007). 

Dillenbourg, Schneider e Synteta (2002) destacam como característica relevante de um AVA o potencial para apoiar a interação social. Dentre os recursos que oferecem oportunidades para que isso se realize, o fórum é uma ferramenta que promove a atividade colaborativa, porque os participantes contribuem, na maioria das vezes, com o intuito de
atingir o consenso ou uma definição sobre um tema (SOUZA, 2012); além disso, estimulam o diálogo, a comunicação e a socialização (OESTERREICH; MONTOLI, 2010). 

Existem várias ferramentas que auxiliam no uso de ambiente virtual de aprendizagem uma que se destaca é o Moodle, Moodle é uma plataforma de aprendizagem concebida para proporcionar aos educadores, administradores e alunos um único sistema robusto, seguro e integrado para criar ambientes de aprendizagem personalizados. (Moodle, 2017)

O Moodle pode ser customizado e receber a identidade visual da instituição que o está utilizando. Por meio do Moodle, os professores têm a possibilidade de criar atividades individuais e coletivas, permitindo a interação com e entre os estudantes de forma síncrona e assíncrona. Além disso, os professores podem utilizar o ambiente para organizar os materiais educacionais e disponibilizar informações e orientações referentes à sua disciplina (Carvalho, 2008).

Historicamente o desenvolvimento da EAD acompanhou a evolução dos recursos tecnológicos, Mari (2011) citando Moore e Kearsley, destaca as 5 gerações da Ead:
\begin{itemize}
    \item Primeira geração: desenvolvida através dos estudos por correspondência;
    \item Segunda geração: caracterizada pelos cursos transmitidos por rádio e televisão;
    \item Terceira geração: utilizava como recurso as diversas tecnologias da comunicação;
    \item Quarta geração: utiliza como recurso a teleconferência;
    \item Quinta geração: salas virtuais com base na utilização dos computadores na internet.
\end{itemize}
O avanço das TICs tem possibilitado a criação de ambientes virtuais de aprendizagem, potencializado pelos recursos da internet.

\section{Educação Inclusiva de Deficientes Visuais}
\lettrine{O} interesse pelo ensino inclusivo no Brasil tem sido crescente nos últimos anos, um direito das crianças e dos adolescentes com necessidades específicas à educação, o qual tem sido garantido desde a Declaração Universal dos Direitos Humanos, inerentes às condições físicas, intelectuais e sociais que a criança possua (UNESCO, 1994).

Conhecida também como cegueira possui algumas causas distintas em seres humanos. Segundo Domingos (2007), algumas das causas da deficiência visual em seres humanos são: retinopatia causada pela imaturidade da retina, catarata congênita, glaucoma, diabetes, traumas, entre outras causas não tão frequentes. Ainda de acordo com Domingos (2007), a perda total da visão pode ocorrer desde o nascimento ou em algum momento da vida através das causas acima citadas.

De acordo com Conde (2006), os portadores de deficiência visual podem ser divididos em dois grupos, classificados da seguinte forma: pessoas com cegueira e pessoa com baixa visão ou com visão subnormal.

A construção de práticas pedagógicas inclusivas, que promovam o acesso aos serviços e recursos pedagógicos e de acessibilidade, viabilizam a superação da discriminação e da segregação no contexto das Instituições de Ensino. De acordo com Saviani (2008), a partir do processo de democratização da educação se evidencia o paradoxo inclusão/exclusão, quando os sistemas de ensino universalizam o acesso, mas continuam excluindo indivíduos e grupos considerados fora dos padrões homogeneizadores da escola. "A construção de uma sociedade de plena participação e igualdade tem como um de seus princípios a interação efetiva de todos os cidadãos. Nesta perspectiva é fundamental a construção de políticas de inclusão para o reconhecimento da diferença e para desencadear uma revolução conceitual que conceba uma sociedade em que todos devem participar, com direito de igualdade e de acordo com suas especificidades". (Conforto `I\&'  Santarosa, 2002).

Conforme Borges (1996) "uma pessoa cega pode ter algumas limitações, as quais poderão trazer obstáculos ao seu aproveitamento produtivo na sociedade". Ele aponta que grande parte destas limitações pode ser eliminada através de duas ações: uma educação adaptada a realidade destes sujeitos e o uso da tecnologia para diminuir as barreiras.

\section{Considerações finais}
\lettrine{N}{o} que tange a deficiência visual, a importância dos Ambientes Digitais é inquestionável. De acordo com Campbell "desde a invenção do Código Braille em 1829, nada teve tanto impacto nos programas de educação, reabilitação e emprego quanto o recente desenvolvimento da Informática para os cegos" (CAMPBELL, 2001).
\chapter{Trabalhos Relacionados} \label{chap:Trabalhos}
\lettrine{O} uso de tecnologias na educação está cada vez maior no mundo todo, principalmente por influência da internet, 
é fácil conectar computadores, até mesmo de dispositivos móveis como celulares para  conseguir qualquer conteúdo educativo ou 
informação importante. Contudo, algumas pessoas ainda encontra dificuldade na utilização destas tecnologias, do qual se destaca 
pessoas com necessidades visuais.

Para esse público específico a quantidade de plataforma que está preparada para receber este tipo de usuário ainda é limitada, e no 
ramo da educação a distância o impacto é maior.
Alguns autores buscam constantemente soluções para incluir este tipo de usuário, neste trabalho algumas soluções já propostas
serão relembradas.

%%%%%%%%%%%%%%%%%%%%FAZER%%%%%%%%%%%%%%%%%%%%%%%%
%%
%% \section{Revisão Sistemática da Literatura}
%% \lettrine{P}{ara} FAZER...
%%
%% \section{Estado da arte}
%% \lettrine{P}{ara} FAZER...
%%
%% \section{Taxonomias}
%% \lettrine{P}{ara} FAZER...
%%
%% \section{Síntese dos trabalhos relacionados}
%% \lettrine{P}{ara} FAZER...
%%
%%%%%%%%%%%%%%%%%%%%%%%%%%%%%%%%%%%%%%%%%%%%%%%%%

\section{Ambientes virtuais de aprendizagem}
\lettrine{A} vasta facilidade de acesso a internet e da necessidade de ambientes educacionais que suportem 
adequadamente as atividades dos alunos e professores, a Web, se apresenta como um recurso pronto para atender 
parte das expectativas dos entusiastas da área de Ambientes virtuais de aprendizagem.

Para se criar o projeto de qualquer sistema acessível, faz-se necessária a interação de mecanismos que disponibilizem o 
conteúdo a partir da conversão de significados, o que pode favorecer a mediação de informações semânticas na utilização 
de ambientes virtuais \cite{Paper2008}.

Na intenção de facilitar para alunos e professores, diversas plataformas tem sido desenvolvida utilizando os recursos 
disponíveis na Web. No contexto das instituições de ensino, alunos e professores percebem, através da disponibilização 
de notas de aula, listas de exercícios e trabalhos na Internet, as vantagens de ter estes recursos presente no processo 
de ensino-aprendizagem, seja ele presencial ou à distância.

\subsection{Moodle}
\lettrine{M}{oodle} (Modular Object Oriented Dynamic Learning Environment) é um Ambiente Virtual de Ensino e Aprendizagem (AVEA), Na educação a distância foi onde o Moodle se tornou muito popular, sua função básica está descrita no manual \cite{Moodle} como “um software desenhado para auxiliar educadores a organizar e gerenciar com facilidade cursos online”. 

O Moodle é um software livre, sua principal função é a de ser um sistema de administração voltado à educação, especialmente dirigido à criação de fórum em ambientes direcionados à aprendizagem.

Fazem uso do Moodle. A utilização de tais ambientes é fundamental ao processo de ensino e aprendizagem dos discentes, mas sabe-se que esse espaço virtual ainda apresenta barreiras no acesso aos conteúdos e 
outras informações, principalmente para o aluno com deficiência. A organização da informação no ambiente virtual, como 
ela chega até o acadêmico e o conteúdo são alguns exemplos de que, além da estrutura do Moodle, a ação docente 
também é imprescindível neste processo \cite{FIALHO}.

Dentre os principais benefícios do uso do Moodle, podem-se destacar:
\begin{itemize}
  \item{É gratuito, isto é, qualquer instituição ou mesmo pessoa física pode efetuar seu download e instalá-lo, sem custos;}
  \item{É open source, o que permite o desenvolvimento de soluções customizadas a partir do mesmo;}
  \item{Adota o padrão Scorm, utilizado em objetos de aprendizagem e outras tecnologias para educação;}
  \item{Apresenta uma grande comunidade de usuários, o que permite a interação e troca de experiências entre os mesmos.}
\end{itemize}

Um projeto que possui bastante citações e tem foco no Moodle e acessibilidade é 
o Easy \cite{rezende2005abaco}, produto gerado a partir da tese de mestrado do professor André Luiz Andrade, no qual usa os dados do Moodle como base e cria um nova interface totalmente voltada para os deficientes visuais.

\section{Acessibilidade na educação}
\lettrine{D}{iversos} trabalhos vêm sendo realizados, em diferentes linhas de pesquisa, com o objetivo de auxiliar pessoas cegas a 
navegar na Web. Alguns destes trabalhos procuram identificar o impacto das diretrizes de acessibilidade propostas pelo 
W3C no desenvolvimento de sites acessíveis, assim como a eficácia dessas diretrizes frente aos problemas encontrados 
pelos usuários cegos \cite{4ce5a8d646ac449e98d079b8f6c3a7c5} \cite{Power2012}.

Outras pesquisas têm como objetivo identificar o comportamento dos usuários cegos ao navegar na Web. 
\cite{Vigo2013} analisaram como os usuários cegos se comportam frente a diversas situações problemáticas 
encontradas ao navegar na Web, com o objetivo de possibilitar uma melhoria nos métodos existentes de avaliação 
e modelagem Web. \cite{Voykinska2016} exploram as motivações, desafios, interações e experiências das 
pessoas cegas ao interagir com conteúdos visuais em redes sociais.

Diversos trabalhos abordam questões específicas para a acessibilidade na Web, como por exemplo, 
a acessibilidade de vídeos online com base em anotações de vídeo e enriquecimento de áudio composto por 
síntese de voz e ícones sonoros \cite{Encelle2011} e a exploração de dados georreferenciados 
por meio de coordenadas de mapas, sons não textuais e saída de voz \cite{Zhao2008}.

Por fim, algumas pesquisas adotam o conceito de que as interfaces projetadas para usuários com 
visão podem não se adequar de maneira satisfatória aos usuários cegos, mesmo que em conformidade 
com as diretrizes de acessibilidade. Assim, são encontrados trabalhos na literatura com o enfoque 
do design elaborado especificamente para as pessoas cegas. \cite{Press2008} desenvolveram uma 
interface do usuário textual (Enhanced Text user Interface - ETI) como uma alternativa 
às interfaces gráficas (GUI). Para avaliar essa nova interface, foi realizado um experimento com 39 
usuários cegos, o qual apresentou um aumento de velocidade na tarefa de busca de conteúdo. Porém, a 
ETI não apresentou melhora na tarefa de navegação, possivelmente devido a problemas relacionados à 
falta de classificação do conteúdo.

Em uma outra abordagem, denominada ABBA \cite{Fayzrakhmanov2010}, a página Web é transformada 
em um modelo semântico formal multi-axial, em que os eixos oferecem meios para serializar o 
documento de acordo com diferentes visões semânticas. Com isso, uma pessoa cega pode navegar 
pelo conteúdo e ir diretamente aos eixos desejados, sendo direcionada para as partes relevantes 
da página. Porém, ainda em desenvolvimento, o ABBA possui um conjunto limitado de eixos desenvolvidos, 
e alguns desses eixos são definidos apenas manualmente.

O conceito de uma solução técnica para filtrar informações redundantes e irrelevantes durante a navegação 
de pessoas cegas na Web é proposta por \cite{Steiner2015}. Esta solução é baseada em um 
algoritmo que analisa o conteúdo HTML das páginas Web, compara os elementos da página com os elementos das 
páginas visitadas anteriormente e com os elementos de um banco de dados compartilhado, com objetivo 
de filtrar informações que não são necessárias aos usuários do leitor de tela. 
\cite{Authors2016} apresentaram um serviço de proxy que adapta as páginas Web e as apresenta de uma 
forma usável aos usuários com deficiência visual. O proxy VIPaware apresenta um serviço que recupera as 
principais informações a partir de qualquer página da Web, analisa o HTML da página solicitada e cria 
um DOM (Document Object Model), que é tratado para eliminar elementos inacessíveis.

A W3C, por meio da WAI (Web Accessibility Initiative), trabalha em diversos padrões
e recomendações que têm por objetivo melhorar a acessibilidade de sites. No entanto,
por focar seus esforços em conformidade com suas recomendações, o W3C promove uma
visão mais técnica e mensurável da acessibilidade. Porém, nem sempre ao atingir um
bom nível de acessibilidade segundo esses padrões, um site ou aplicação possui uma boa
usabilidade \cite{Petrie2007}.

\section{Técnicas de inteligência artificial}
\lettrine{A} introdução de técnicas de inteligência artificial nestes ambientes tem diversas finalidades, 
sendo alguma delas;  acompanhamento do aluno, modelos do processo de ensino-aprendizagem melhores; maior 
possibilidade de manter o foco e concentração do aluno.

Atualmente existem diversas técnicas de inteligência artificial, e vários artigos mostram formas de utilização, 
voltada para o ambiente da educação, como MultiAgentes, RBC, entre outros.

A tecnologia de agentes também tem sido aplicada no projeto de outros tipos de sistemas educacionais, 
com alguma ênfase no uso de agentes em Ambientes Virtuais de Aprendizagem (AVA) e no suporte ao uso de OA. 
Como exemplos recentes de aplicações na área de AVA destacam-se os trabalhos \cite{Arias}  e \cite{Campana2008} que 
apresentam propostas arquiteturas multiagente para AVA. No contexto da aplicação da tecnologia de agentes para OA, 
se destaca a proposta dos objetos de aprendizagem inteligentes 
(ILO-Intelligent Learning Objects) \cite{Gomes2004} \cite{Bavaresco2009}.

A Web Semântica busca descrever o conteúdo dos recursos da Web atual, com o objetivo de dar suporte, 
tanto para agentes humanos quanto para agentes artificiais, no processamento de informações. 
Diante desta perspectiva, a comunidade de Inteligência Artificial e Educação tem demonstrado 
interesse neste ramo de pesquisa, com o ¸ intuito de evoluir os sistemas educacionais atuais,
criando ambientes que sejam adaptativos e semânticos.

A Web Semântica, oferece diversas melhorias no contexto de Sistemas Educacionais Baseados na Web, 
contribuindo para o aumento da qualidade da aprendizagem, criando assim os Ambientes Educacionais 
Baseados na Web Semântica.

A aplicação de tecnologias relacionadas a Web Semântica no projeto de ambientes e sistemas educacionais é um fenômeno atual \cite{Isotani} \cite{Proceedings2007}. Apesar disso, a tecnologia de engenharia de ontologias, 
tem se mostrado útil na concepção de vários tipos de ambientes educacionais, incluindo, entre outros, autoria 
de conteúdos \cite{Isotani} \cite{Isotani2008}, ambientes web \cite{Silva2009} \cite{Bittencourt2009} e 
modelos educacionais formais \cite{Hayashi2009}. Nesse contexto, ontologias são tipicamente empregadas para 
definir as propriedades dos elementos e entidades relativas ao sistema educacional. Há uma tendência a seguir 
a estruturação dos ITS e dividir as ontologias educacionais em três tipos \cite{Silva2009}: a) ontologias 
para o domínio de ensino, b) ontologias sobre métodos pedagógicas, e c) ontologias a respeito do modelo de aluno. 
Alternativamente, existem propostas bastante detalhadas de ontologias que integram aspectos dos três tipos citados acima, 
sendo capazes, por exemplo de modelar as várias propriedades de um processo de aprendizagem em diversos níveis de 
granularidade \cite{Proceedings2007} ou descrever processos de aprendizagem colaborativa \cite{Hayashi2009}.

\section{Considerações finais}
\lettrine{T}{odos} os trabalhos abordam formas e ferramentas diferentes para a inclusão do deficiente visual no mundo da educação. 
Todas as técnicas tem seus méritos, e no momento da criação resolveu o problema daquele momento, porém apenas incluí-los 
não auxílio na evolução dele na sociedade. Hoje temos várias formas de trabalhar e progredir nos estudos, com vários métodos 
que auxiliam o aluno no avanço da barreira do conhecimento, mas para as pessoas com necessidades especiais ainda é limitado 
os recursos.

Mesmo quando existe uma ferramenta que auxilie a usar a ferramenta de ambiente virtual, como exemplo, um leitor de 
tela ou uma ferramenta de áudio, a parte da educação, muitas vezes, é deixada de lado.
\chapter{Proposta} \label{chap:Proposta}
\lettrine{O} uso de tecnologias na educação está cada vez maior no mundo todo, principalmente por influência da internet, é fácil conectar computadores e até mesmo de dispositivos móveis como celulares para conseguir qualquer conteúdo educativo ou informação importante. Contudo, algumas pessoas ainda encontra dificuldade na utilização destas tecnologias, do qual se destaca pessoas com necessidades visuais.

\section{Proposta}
\lettrine{A} Proposta consiste em criar um tutor que auxilie no uso da plataforma para deficientes visuais.

Para a inserção do tutor será criada uma interface totalmente voltada para deficientes visuais com os recursos de reconhecimento de voz, leitor de texto, entre outros. Dentre os diversos LMS(Learning Management Systems) existentes, foi selecionado o Moodle, por ser um dos mais utilizados segundo Coelho(2011) \citep{coelho2011acessibilidade}.


\section{Trabalhos Relacionados}
\lettrine{U}{m} trabalho que inspirou a criação de uma interface totalmente voltada para deficientes visuais e usando a base do Moodle como referência foi o trabalho do André Luiz Andrade \citep{rezende2005abaco}. O trabalho do Kraus, Helton e Anita \citep{kraus2007desenvolvimento} trouxeram a ideia da criação de um chatterbot com a técnica de RBC. O trabalho do Azeta, Ayo, Atayero and Ikhu-Omoregbe \citep{azeta2009case} trouxe a ideia de juntar a técnica de RBC para auxiliar deficientes visuais.

\section{Descrição da proposta}
\subsection{Interface}
\lettrine{P}{essoas} que não possuem necessidades visuais podem utilizar a plataforma padrão normalmente, apenas pessoas com necessidades especiais utilizarão a nova interface, assim não interferindo no uso e interação daqueles já habituados com o sistema anterior como exemplificado na imagem abaixo.

\begin{figure}[htbp!]
\centering
\includegraphics[scale=0.7]{img/uso.png}
\caption{Exemplo do uso da interface}
\label{fig:uso}
\end{figure}

A interface irá acessar os dados da base do Moodle, assim para os professores e alunos que não possuem deficiência visual não haverá diferença, excluindo a necessidade de uma nova aprendizagem para todos os envolvidos.

\subsection{Tutor}
\lettrine{N}{a} utilização do sistema, o tutor irá fazer todo o trabalho de auxílio para o deficiente visual, suas tarefas principais se resumem em auxiliar no uso da plataforma e ajustar o conteúdo para o usuário.

\begin{figure}[htbp!]
\centering
\includegraphics[scale=0.6]{img/tutor.png}
\caption{Trabalho do Tutor}
\label{fig:tutor}
\end{figure}

\subsection{Ações do Tutor}
\lettrine{O} tutor está dividido em dois agentes, o primeiro é um chatterbot que resolver as questões de uso da plataforma utilizando a técnica de raciocínio baseado em casos(RBC), o segundo agente está integrado a plataforma e conforme o usuário utiliza, novos conteúdos são sugeridos também baseado na técnica de RBC.

\begin{figure}[htbp!]
\centering
\includegraphics[scale=0.6]{img/Agente.png}
\caption{Ações dos agentes}
\label{fig:agente}
\end{figure}

\subsection{Raciocínio Baseado em Casos(RBC)}
\lettrine{A} ideia principal do RBC é resolver cada novo problema lembrando das soluções de situações anteriores similares. A definição clássica de um sistema RBC foi elaborada por Reisbeck e Schank (1989) \citep{Riesbeck}: “Um sistema RBC resolve problemas, adaptando soluções que foram utilizadas para resolver problemas anteriores”.

Um exemplo seria, uma pessoa escolhe comprar um objeto, e com base nas experiências passadas decide qual será o melhor para não passar por problemas já vividos com outros objetos.

\begin{figure}[htbp!]
\centering
\includegraphics[scale=0.6]{img/RBC.png}
\caption{Ações do Tutor}
\label{fig:Ciclo RBC}
\end{figure}

\chapter{Abordagem} \label{chap:Abordagem}
\lettrine{P}{ara} FAZER...
\chapter{Resultados Parciais} \label{chap:Resultado}
\lettrine{P}{ara} FAZER...
\chapter{Conclus�es} \label{chap-Concl}

Neste trabalho foram considerados problemas de escalonamento com penalidades de antecipa��o e atraso em ambiente monoprocessado e ambiente de m�quinas paralelas id�nticas, com tarefas independentes, de tempos de processamento arbitr�rios e pesos distintos ($P$ $||$ $\sum{\alpha_{j}E_{j}}$ $+ \sum{\beta_{j}T_{j}}$ na nota��o de $3$ campos). A fundamenta��o te�rica envolvendo Otimiza��o Combinat�ria, com �nfase em teoria de escalonamento, programa��o inteira e m�todos heur�sticos, juntamente com um detalhamento aprofundado do problema de escalonamento em enfoque no trabalho, envolvendo defini��es, nota��o cl�ssica, discuss�o sobre estrutura de solu��o com empates e simetria, formula��es matem�ticas existentes e trabalhos relacionados. Foram apresentadas quatro abordagens algor�tmicas aproximadas e uma exata para resolu��o do problema.

Dentre as estrat�gias aproximadas, foram considerados os m�todos de \textbf{busca local iterada}, algoritmo inicialmente proposto por Rodrigues et al. \cite{RosianeArthurEduardoMarcus:2008} e que foi adaptado para o problema considerado neste trabalho, o m�todo de \textbf{reconex�o de caminhos com busca local iterada}, cujo objetivo � encontrar melhores solu��es no espa�o de busca formado entre dois �timos locais e, o terceiro m�todo trata-se de um m�todo de otimiza��o global com busca m�ltipla baseada em popula��o que foi desenvolvido atrav�s do \textbf{algoritmo gen�tico com reconex�o de caminhos} onde, a cada itera��o, � gerada uma nova popula��o com solu��es cada vez melhores e diversificadas.

No quarto m�todo proposto tamb�m foi utilizada a ideia de busca m�ltipla baseada em popula��o, que foi proposto atrav�s do \textbf{algoritmo gen�tico} utilizando \textbf{busca local} com \textbf{reconex�o de caminhos}, cujo objetivo foi de explorar cada vez mais o espa�o de busca entre duas solu��es �timas locais afim de convergir rapidamente para melhores solu��es. A quinta e �ltima estrat�gia algor�tmica apesentada envolveu o ILS com busca m�ltipla baseada em um conjunto de solu��es, onde o objetivo desta estrat�gia foi de verificar se o algoritmo ILS consegue atingir melhores solu��es para inst�ncias de tamanho maior, considerando um conjunto de solu��es diversificadas a cada itera��o.

O m�todo exato considerado neste trabalho foi algoritmo \textit{branch-and-cut} do CPLEX, onde foram consideradas duas formula��es matem�ticas, uma formula��o para o ambiente monoprocessado, sem tempo ocioso entre as tarefas, proposta por Tanaka et al.~\cite{Tanaka:2009:EAS:1644424.1644462}, e a outra formula��o para o ambiente de m�quinas paralelas, formula��o cl�ssica de escalonamento que, atrav�s dos experimentos computacionais apresentados, essa formula��o n�o trata o tempo ocioso em algumas inst�ncias. A implementa��o contou com a utiliza��o da biblioteca UFFLP para C++ onde, a partir de uma inst�ncia de teste, foi poss�vel gerar o modelo matem�tico a ser executado pelo CPLEX.

Uma an�lise comparativa emp�rica mostrou que a estrat�gia baseada em popula��o do algoritmo gen�tico com busca local e reconex�o de caminhos entre dois �timos locais foi o melhor e mais promissor m�todo proposto, atingindo todos os �timos da literatura para o ambiente de monoprocessado. Os m�todos apresentados tamb�m s�o adapt�veis ao ambiente de m�quinas paralelas, onde � poss�vel atingir boas solu��es em um tempo razo�vel. Infelizmente ainda n�o existem \textit{benchmarks} na literatura para m�quinas paralelas para compara��o com os nossos resultados, mais este trabalho apresentou uma compara��o dos resultados dos m�todos aproximados e exatos a fim de se tenha uma \textbf{refer�ncia de qual seria a melhor solu��o} ou \textbf{solu��o �tima} para o problema envolvendo \textbf{ambiente multiprocessado}, espera-se tamb�m que estes resultados possam ser utilizados como \textbf{refer�ncia em outras pesquisas} para fins de compara��o com outros m�todos.

Os resultados obtidos a partir da execu��o do m�todo exato mostraram que, quando o mesmo consegue terminar, ou seja, encontrar a melhor solu��o, a execu��o do \textit{branch-and-cut} do CPLEX � mais r�pida para m�quinas paralelas, ao contr�rio dos m�todos aproximados propostos. Isso se deve ao fato de que o tempo total de processamento tende a diminuir a medida que o n�mero de m�quinas dispon�veis aumenta, pois, para os m�todos aproximados, considera-se o custo de distribuir as tarefas nas m�quinas utilizando regras de despacho toda vez que uma solu��o tiver que ser avaliada. Mas, apesar disso, os m�todos aproximados s�o extremamente promissores para o problema, apresentando bons resultados em um razo�vel espa�o de tempo, al�m do mais, o tempo de processamento dos m�todos aproximados apresentados podem ser minimizados atrav�s da combina��o de novas estrat�gias de solu��o.

\section{Confer�ncias e publica��es}

Os trabalhos apresentados/publicados est�o listados abaixo em ordem cronol�gica do mais recente ao mais antigo:

 	\begin{itemize}
 	 \item  \textbf{Em processo de publica��o para a revista \textit{Expert Systems with Applications}}.
 	 \vspace{-0.3cm}
                \begin{itemize}
                   \item \textit{Single and multi-start methods based on local search and path-relinking technique for earliness-tardiness parallel machine scheduling problems}.
                \end{itemize}
     \item  \textbf{Trabalho no GECCO 2013 - Genetic and Evolutionary Computation Conference}.
     \vspace{-0.3cm}
                \begin{itemize}
                   \item \textit{A hybrid genetic algorithm with local search approach for E/T scheduling problems on identical parallel machines}~\cite{amorim-dias-rodrigues-uchoa:2013}.
                \end{itemize}
     \item  \textbf{Trabalho no WoPI 2012 - II Workshop de Pesquisa em Inform�tica}.
     \vspace{-0.3cm}
                \begin{itemize}
                   \item \textit{Minimiza��o da Antecipa��o e Atraso em Escalonamento de Processos Produtivos Usando M�quinas Paralelas}~\cite{amorim-rodrigues3:2012}.
                \end{itemize}
     \item  \textbf{Trabalho no EURO-INFORMS MMXIII - 26th European Conference on Operational Research}.
     \vspace{-0.3cm}
                \begin{itemize}
                   \item \textit{Algorithmic strategies to single and parallel machine scheduling problems with earliness-tardiness penalties}~\cite{amorim-rodrigues2:2013}.
                \end{itemize}
     \item  \textbf{Trabalho no EURO 2012 - $25^{th}$ European Conference on Operation Research}.
     \vspace{-0.3cm}
                \begin{itemize}
                   \item \textit{MIP models and algorithms for earliness/tardiness scheduling problems on parallel machines}~\cite{amorim-dias-rodrigues2:2012}.
                \end{itemize}
     \item  \textbf{Trabalho no CLAIO/SBPO 2012 - Congreso Latino-iberoamericano de Investigaci�n Operativa / Simp�sio Brasileiro de Pesquisa Operacional}.
     \vspace{-0.3cm}
                \begin{itemize}
                   \item \textit{ILS com reconex�o de caminhos entre �timos locais para um problema cl�ssico de escalonamento com antecipa��o e atraso}~\cite{amorim-dias-rodrigues:2012}.
                \end{itemize}
 	 \item  \textbf{Participa��o na ELAVIO 2012 - XVI Escuela Latinoamericana de Verano en Investigaci�n Operativa}.
 	 \vspace{-0.3cm}
                \begin{itemize}
                   \item \textit{Problemas cl�ssicos de escalonamento sob restri��es de antecipa��o e atraso}~\cite{amorim-rodrigues:2012}.
                   \item Participa��o na din�mica de grupo.
                \end{itemize}
    \end{itemize}

Al�m dos eventos e publica��es supracitadas, foram realizadas visitas t�cnicas em institui��es parceiras do Rio de Janeiro, incluindo o Departamento de Engenharia de Produ��o da Universidade Federal Fluminense, o Programa de Engenharia de Sistemas e Computa��o da COPPE/Universidade Federal do Rio de Janeiro e a sede do Instituto Nacional de Metrologia, Qualidade e Tecnologia (Inmetro), onde houve um encontro t�cnico com exposi��es tanto de pesquisas realizadas neste trabalho quanto de membros da institui��o.


\section{Trabalhos futuros}

Para trabalhos futuros envolvendo problemas de escalonamento com penalidades de antecipa��o e atraso, considera-se importante serem  propostas formula��es matem�ticas em programa��o inteira mais robustas para tais problemas, de tal forma a serem desenvolvidos m�todos exatos mais eficientes. M�todos h�bridos, envolvendo a resolu��o de formula��es relaxadas para o problema em conjunto com m�todos heur�sticos, � um caminho interessante a seguir, no intuito de se obter solu��es �timas com tempos de execu��o mais eficientes. 

Um aprofundamento no estudo e tratamento de solu��es com empate e sim�tricas tamb�m faz-se necess�rio, de modo que procedimentos algor�tmicos que sejam capazes de detectar tais solu��es equivalentes sejam elaborados. Assim, tais solu��es equivalentes n�o precisam ser consideradas durante o processo de busca por melhores solu��es, permitindo uma redu��o significativa do tempo de processamento do algoritmo. Por outro lado, � interessante tamb�m investigar novos ambientes de processamento como m�quinas de prop�sito geral (\textit{mpm}) e diferentes tipos de restri��es, tais como: tempos de prepara��o (\textit{setups}), datas de t�rmino iguais (\textit{common due dates}) e preemp��o (\textit{pmtn}).


\bibliographystyle{alpha-ime}
\label{bib:begin}
\bibliography{references}
\label{bib:end}
\end{document}