\documentclass[mscexam,pdftex]{ppgi-ufam}
\usepackage[brazil]{babel}
\usepackage[utf8]{inputenc}
\usepackage[T1]{fontenc}
\let\newfloat=\undefined
\usepackage{enumerate}
\usepackage{amsmath}
\usepackage{amsthm}
\usepackage{amssymb}
\usepackage{array}
\usepackage{longtable}
\usepackage{caption}
\usepackage{listings}
\usepackage{pslatex}
\usepackage{setspace}
\usepackage[all]{hypcap}
\usepackage[nottoc]{tocbibind}
\usepackage[square,sort,nonamebreak,comma]{natbib}
\usepackage{epstopdf}
\usepackage{scalefnt}

\newcommand{\tab}{\hspace*{0.5em}}
\usepackage{float}
\usepackage{comment}
\usepackage{booktabs}
\usepackage{lscape}
\addto{\captionsbrazil}{
    \renewcommand{\bibname}{Refer\^{e}ncias}
}

\makelosymbols
\makeloabbreviations
\hypersetup{hidelinks}
\usepackage{lettrine} 

\begin{document}

\title{Tutor inteligente com raciocínio baseado em casos para deficientes visuais na plataforma moodle}
\foreigntitle{Intelligent tutor with case-based reasoning for the visually impaired on the moodle platform}
\author{Walter Nascimento}{Barroso}
\advisor{Profa.}{Elaine Harada Teixeira de}{Oliveira}{D.Sc.}{Instituto de Computação}{Universidade Federal do Amazonas}
\examiner{Prof.}{xxx}{D.Sc.}{Instituto de Computação}{Universidade Federal do Amazonas}
\examiner{Profa.}{xxx}{D.Sc.}{Instituto de Computação}{Universidade Federal do Amazonas}
\examiner{Prof.}{xxx}{D.Sc.}{Instituto de Computação}{Universidade Federal do Amazonas}
\examiner{Prof.}{xxx}{D.Sc.}{Instituto de Computação}{Universidade Federal do Amazonas}
\department{IComp}
\date{03}{2017}

\keyword{Raciocínio baseado em casos}
\keyword{Educação}
\keyword{E-learning}
\keyword{Educação a distância}
\keyword{Deficientes visuais}
\keyword{Ensino online}
\keyword{Inteligência artificial}

\maketitle


\dedication{A Deus, \\ aos professores, \\ aos colegas de curso e \\ aos meus familiares.}

\chapter*{Agradecimentos}
Agradeço a Deus.


\begin{abstract}
Esta pesquisa investiga ... .
\linebreak
\linebreak
\textbf{Palavras-chave:} Raciocínio baseado em casos, Educação, E-learning, Educação a distância, Deficientes visuais, Ensino online, Inteligência artificial.
\end{abstract}

\begin{foreignabstract}
This research investigates ... .
\linebreak
\linebreak
\textbf{Keywords:} Case-based Reasoning, Education, E-learning, Distance Education, Visually impaired, Online Teaching, Artificial Intelligence.
\end{foreignabstract}

\setcounter{tocdepth}{1}

\tableofcontents
\listoffigures
\listoftables
    
\mainmatter

\chapter{Introdução} \label{chap:Introdução}
\lettrine{T}{odo} mundo tem o direito à educação, e hoje obter informação é algo tão simples, porém mesmo sendo algo simples para a maioria das pessoas, existe as pessoas com necessidades especiais que na maiorias das vezes é deixado de lado, e dentre as várias necessidades existentes, a deficiência visual é a menos tratada no mundo da informação compartilhada, pois é a mais difícil, tendo em vista que todo o sistema tem que ser de fácil acesso por um leitor de tela.

Com o passar do tempo surgiram várias formas de concluir cursos a distância, hoje é uma das principais formas de estudo, porém pessoas com deficiência visual ainda tem grande dificuldade em utilizar todos os recursos das plataformas EAD Um grande obstáculo para todos os estudante é a busca por conhecimento.

Raciocínio Baseado em Casos , ou Case Based Reasoning (CBR), é uma técnica que busca a solução para uma situação atual através da recuperação e adaptação de soluções passadas semelhantes, dentro de um mesmo domínio do problema. O sistema é capaz de localizar e encontrar partes de casos que não se adequem ao problema, criando um novo caso para uso posterior.

Moodle é uma plataforma de aprendizagem concebida para proporcionar aos educadores, administradores e alunos um único sistema robusto, seguro e integrado para criar ambientes de aprendizagem personalizados.

As pessoas com deficiência visuais que estão fora do sistema educacional e, consequentemente, sem acesso à cultura na vida adulta, podem encontrar dificuldades para conquistar a sua independência pessoal, sendo assim, pouco ou nada contribuirão ao país.

Aplicando RBC em uma plataforma de ensino online (Moodle) é possível transformar um ambiente simples em um ambiente inteligente que aprende com 1o conhecimento do aluno e auxiliar na escolha de quais níveis avançar, baseado em casos já existentes e adicionando novos casos a cada interação com o sistema.


\section{Contexto e descrição do problema}
\lettrine{U}{sar} um sistema, mesmo para uma pessoa com todos os sentidos ainda sim é algo complicado, e por tanto para alguém que possui uma necessidade especial tão difícil como a deficiência visual e uma barreira bem maior conseguir usar e usufruir de todos os recursos do sistema, tornando o uso de um ambiente virtual quase impossível de ser usado.

Hoje temos diversos recursos de leitores de tela que facilitam o uso de diversas ferramentas, mas ainda assim não é o suficiente para trazer toda a experiencia necessária para concluir com qualidade cursos a distancia, plataformas como o Moodle possui diversos plugins para auxiliar na necessidade dos alunos, oferecendo leitores de telas e dicas.

Mas mesmo com todos os recursos disponíveis, varias funcionalidades passam despercebidas.

\section{Motivação}
\lettrine{P}{ara} FAZER...

\section{Objetivos}

\subsection{Objetivo Geral}
\lettrine{P}{ropor} um tutor inteligente que use Raciocínio Baseado em Casos, para auxiliar
no uso da plataforma Moodle, específico para deficientes visuais.

\subsection{Objetivos específicos}
\lettrine{O}{s} objetivos específicos incluem:
\begin{itemize}
    \item Fazer um levantamento do que já existe de acessibilidade em Ambientes Virtuais de Aprendizagem para pessoas com deficiência visual;
    \item Efetuar reconhecimento de um padrão em uma sequência de quadros;
    \item Desenvolver técnicas para seleção do melhor caminho para aprendizagem
    de cada aluno;
\end{itemize}

\section{Justificativa}
\lettrine{P}{ara} FAZER...

\section{Questões de Pesquisa}
\lettrine{P}{ara} FAZER...

\section{Metodologia}
\lettrine{P}{ara} FAZER...

\section{Cronograma}
\lettrine{P}{ara} FAZER...

\section{Organização do documento}
\lettrine{P}{ara} FAZER...
\chapter{Fundamentação Teórica} \label{chap:Fundamentação}
\lettrine{N}{este} capítulo são apresentados os conceitos fundamentais levantados através de uma revisão sistemática da literatura (RSL) e complementado por uma pesquisa exploratória. 

Desta forma, são apresentados os conceitos sobre raciocínio baseado em casos, ambiente virtual de aprendizagem e educação inclusiva de deficientes visuais.

\section{Raciocínio Baseado em Casos}
\lettrine{A} idéia principal do RBC é resolver cada novo problema lembrando das soluções de situações anteriores similares.

A definição clássica de um sistema RBC foi elaborada por Reisbeck e Schank \cite{Riesbeck}: “Um sistema RBC resolve problemas, adaptando soluções que foram utilizadas para resolver problemas anteriores”.

Um exemplo seria, uma pessoa escolhe comprar um objeto, e com base nas experiências
passadas decide qual será o melhor para não passar por problemas já vividos com outros
objetos.

\cite{WANGENHEIM2013} Afirmam que dá-se o nome de Raciocínio Baseado em Casos (RBC), à técnica de Inteligência Artificial, como um conjunto de atividades que auxilia na resolução de problemas, propondo soluções incontestavelmente utilizadas e documentadas, ao recuperar e adaptar experiências passadas – chamadas casos – armazenadas em uma Base de Casos. Um novo caso é resolvido com base na adaptação de solução de casos similares já conhecidos \cite{WANGENHEIM2013}.

O ciclo de funcionamento de um sistema de RBC é composto por quatro etapas de execução, conhecida como 4R, conforme definido por \cite{aamodt1994case}, explicado abaixo e ilustrado pela Figura 1.
\begin{itemize}
    \item Recuperação: a partir da apresentação ao sistema de um novo problema é feita a recuperação na base de casos daquele mais parecido com o problema em questão. Isto é feito a partir da identificação das características mais significantes em comum entre os casos;
    \item Reuso: a partir do caso recuperado é feita a reutilização da solução associada àquele caso. Geralmente a solução do caso recuperado é transferida ao novo problema diretamente como sua solução;
    \item Revisão: é feita quando a solução não pode ser aplicada diretamente ao novo problema. O sistema avalia as diferenças entre os problemas (o novo e o recuperado), quais as partes do caso recuperado são semelhantes ao novo caso e podem ser transferidas adaptando assim a solução do caso recuperado da base à solução do novo caso;
    \item Retenção: é o processo de armazenar o novo caso e sua respectiva solução para futuras recuperações. O sistema irá decidir qual informação armazenar e de que forma.
\end{itemize}

\begin{figure}[htbp!]
\centering
\includegraphics[scale=0.6]{img/cicloRBC.png}
\caption{Ciclo do Raciocínio Baseado em Casos.}
\label{fig:cicloRBC}
\end{figure}

Conforme Wangenheim \cite{WANGENHEIM2003}, as etapas mais importantes do processo de
desenvolvimento de um sistema RBC são:
\begin{itemize}
    \item Aquisição de Conhecimento;
    \item Representação de Caso;
    \item Indexação;
    \item Recuperação de Casos;
    \item Adaptação de Casos.
\end{itemize}

\subsection{A Aquisição de Conhecimento}
\lettrine{A}{quisição} de conhecimento é uma etapa muito importante no desenvolvimento do sistema de RBC. Conforme Kolodner \cite{Kolodner1993} ela é considerada o componente crítico no desenvolvimento de sistemas de RBC. Consiste na seleção de casos que irão formar uma base de informações (um sistema de banco de dados – uma base de casos) que contenha implicitamente o conhecimento necessário na solução de problemas;

\subsection{Representação de caso}
\lettrine{U}{m} caso é definido pela representação do conhecimento contido em uma experiência vivida que dirige o indivíduo a alcançar seus objetivos \cite{leakecase}. Todo caso é composto por: Problema: descreve o estado do mundo real onde o caso ocorre; e, Solução: contém o estado das soluções derivadas para o problema \cite{watson1998applying}. Pode-se armazenar os casos utilizando os diferentes tipos usados pelos computadores, arquivos, banco de dados, etc.

\subsection{Indexação}
\lettrine{I}{ndexação} é um processo importante e decisivo dentro do RBC, assim como numa base de dados convencional, porém não necessariamente, todas as informações relevantes deverão ser indexadas. As informações indexadas servirão para acelerar o processo de recuperação enquanto que informações não-indexadas podem servir de contextualização para a decisão de reutilização do caso ou em outros aspectos do sistema de informação e não somente na recuperação. \cite{vitorino2009raciocinio} Uma das tarefas da indexação é atribuir pesos às características dos casos, para que possa alcançar a recuperação de casos utilizando o algoritmo de vizinhança. Com esta técnica define peso as características mais importantes.

\subsection{Recuperação de Casos}
\lettrine{O} processo de recuperação de casos \cite{Riesbeck} inicia com uma descrição de problema e finaliza quando um melhor caso for encontrado. O sistema procura na base de casos, o caso mais similar com o novo problema.

\cite{leakecase} coloca que uma característica importante dos sistemas de RBC é possuir alternativas para identificar os casos a fim de conseguir representá-los e indexá-los, garantindo que sejam recuperados os mais úteis para resolver o problema do usuário. Somente consegue-se alternativas para identificar os casos através de procedimentos de comparação e medição de similaridades.

\section{Ambiente Virtual de Aprendizagem}
\lettrine{O}{s} AVA são espaços virtuais que permitem aos sujeitos envolvidos nos processos de ensino, aprendizagem e avaliação a busca por conhecimentos e capacitação \cite{maciel2012ambientes}. São projetados para facilitar o acesso a materiais de aprendizagem e a comunicação dos estudantes entre si e com os professores. Em resumo, refere-se a um espaço no qual ocorre a comunicação pedagógica em processos formativos semipresenciais ou a distância \cite{adell2010ambientes}. Segundo Almeida \cite{almeida2011educaccao}, os ambientes virtuais permitem integrar múltiplas mídias, linguagens e recursos, apresentar informações de maneira organizada e desenvolver interações entre pessoas. Oferecem possibilidades para a criação de espaços educacionais diferenciados que valorizam a participação do aluno de forma mais contextualizada e integrada aos objetivos de aprendizagem. 

Dillenbourg e grupo \cite{dillenbourg2002virtual} destacam como característica relevante de um AVA o potencial para apoiar a interação social. Dentre os recursos que oferecem oportunidades para que isso se realize, o fórum é uma ferramenta que promove a atividade colaborativa, porque os participantes contribuem, na maioria das vezes, com o intuito de atingir o consenso ou uma definição sobre um tema \cite{de2012aprendizagem} além disso, estimulam o diálogo, a comunicação e a socialização \cite{oesterreich2010potencialidades}.

Existem várias ferramentas que auxiliam no uso de ambiente virtual de aprendizagem uma que se destaca é o Moodle, Moodle é uma plataforma de aprendizagem concebida para proporcionar aos educadores, administradores e alunos um único sistema robusto, seguro e integrado para criar ambientes de aprendizagem personalizados \cite{Moodle}.

O Moodle pode ser customizado e receber a identidade visual da instituição que o está utilizando. Por meio do Moodle, os professores têm a possibilidade de criar atividades individuais e coletivas, permitindo a interação com e entre os estudantes de forma síncrona e assíncrona. Além disso, os professores podem utilizar o ambiente para organizar os materiais educacionais e disponibilizar informações e orientações referentes à sua disciplina \cite{carvalho2008lms}.

Historicamente o desenvolvimento da EAD acompanhou a evolução dos recursos tecnológicos, Mari \cite{mari2011avaliaccao} citando Moore e Kearsley, destaca as 5 gerações da Ead:
\begin{itemize}
    \item Primeira geração: desenvolvida através dos estudos por correspondência;
    \item Segunda geração: caracterizada pelos cursos transmitidos por rádio e televisão;
    \item Terceira geração: utilizava como recurso as diversas tecnologias da comunicação;
    \item Quarta geração: utiliza como recurso a teleconferência;
    \item Quinta geração: salas virtuais com base na utilização dos computadores na internet.
\end{itemize}
O avanço das TICs tem possibilitado a criação de ambientes virtuais de aprendizagem, potencializado pelos recursos da internet.

\section{Educação Inclusiva de Deficientes Visuais}
\lettrine{O} interesse pelo ensino inclusivo no Brasil tem sido crescente nos últimos anos, um direito das crianças e dos adolescentes com necessidades específicas à educação, o qual tem sido garantido desde a Declaração Universal dos Direitos Humanos, inerentes às condições físicas, intelectuais e sociais que a criança possua \cite{de1994linha}.

Conhecida também como cegueira possui algumas causas distintas em seres humanos. Segundo Domingos \cite{domingos2008sexualidade}, algumas das causas da deficiência visual em seres humanos são: retinopatia causada pela imaturidade da retina, catarata congênita, glaucoma, diabetes, traumas, entre outras causas não tão frequentes. Ainda de acordo com Domingos \cite{domingos2008sexualidade}, a perda total da visão pode ocorrer desde o nascimento ou em algum momento da vida através das causas acima citadas.

\begin{table}[htbp]
\begin{tabular}{ |p{4.5cm}|p{10cm}| }
\hline
\multicolumn{2}{ |c| }{\textbf{Incapacidade Visual}} \\
\hline
\textbf{Cenário} & \textbf{Barreiras na Web} \\ 
\hline
\multirow{7}{*}{\textbf{\minitab[c]{Perda total da visão em \\ ambos os olhos}}} 
& $\bullet$ Imagens sem texto alternativo. \\
& $\bullet$ Gráficos e imagens complexas indevidamente descritas. \\
& $\bullet$ Imagens dinâmicas sem áudio-descrição ou sem texto complementar. \\
& $\bullet$ Formulários e Tabelas complexas que não permitem uma leitura linear ou perdem o sentido. \\
& $\bullet$ 'Frames' sem nomes ou com nomes imperceptíveis. \\
& $\bullet$ Ferramentas de autor ou browsers que não permitem activação se todos os comandos ou instruções por teclado. \\
& $\bullet$ Ferramentas de autor ou browsers que não utilizam programas ou aplicações com interface normalizado dificultando a leitura e interpretação ao leitor de ecrã. \\
\hline
\multirow{3}{*}{\textbf{\minitab[c]{Visão reduzida, visão \\ pouco nítida ou \\ desfocada, redução do \\ campo de visão}}} 
& $\bullet$ Tamanho de texto pequeno que não permite ampliar. \\
& $\bullet$ Dificuldade de navegação quando o ecrã é ampliado. \\
& $\bullet$ Texto colocado como imagem que pode perder a definição quando é ampliado. \\
\hline
\multirow{3}{*}{\textbf{\minitab[c]{Falta de sensibilidade a \\ algumas cores}}} 
& $\bullet$ Texto destacado apenas pela cor. \\
& $\bullet$ Baixo ou inadequado contraste entre texto e fundo. \\
& $\bullet$ Browsers ou aplicações que não permitem personalização ou não suportam ferramentas de alto. \\
\hline
\end{tabular}
\caption{Incapacidade visual – cenários e barreiras Fonte:(FRANCISCO, 2008, p. 52) \cite{francisco2008contributos}.}
\label{tab:bar}
\end{table}

De acordo com Conde \cite{conde2007definindo}, os portadores de deficiência visual podem ser divididos em dois grupos, classificados da seguinte forma: pessoas com cegueira e pessoa com baixa visão ou com visão subnormal.

A construção de práticas pedagógicas inclusivas, que promovam o acesso aos serviços e recursos pedagógicos e de acessibilidade, viabilizam a superação da discriminação e da segregação no contexto das Instituições de Ensino. De acordo com Saviani \cite{saviani2003pedagogia}, a partir do processo de democratização da educação se evidencia o paradoxo inclusão/exclusão, quando os sistemas de ensino universalizam o acesso, mas continuam excluindo indivíduos e grupos considerados fora dos padrões homogeneizadores da escola. "A construção de uma sociedade de plena participação e igualdade tem como um de seus princípios a interação efetiva de todos os cidadãos. Nesta perspectiva é fundamental a construção de políticas de inclusão para o reconhecimento da diferença e para desencadear uma revolução conceitual que conceba uma sociedade em que todos devem participar, com direito de igualdade e de acordo com suas especificidades" \cite{conforto2002acessibilidade}.

Conforme Borges \cite{borges1996dosvox} "uma pessoa cega pode ter algumas limitações, as quais poderão trazer obstáculos ao seu aproveitamento produtivo na sociedade". Ele aponta que grande parte destas limitações pode ser eliminada através de duas ações: uma educação adaptada a realidade destes sujeitos e o uso da tecnologia para diminuir as barreiras.

\section{Considerações finais}
\lettrine{N}{o} que tange a deficiência visual, a importância dos Ambientes Digitais é inquestionável. De acordo com Campbell \cite{campbell2001trabalho} "desde a invenção do Código Braille em 1829, nada teve tanto impacto nos programas de educação, reabilitação e emprego quanto o recente desenvolvimento da Informática para os cegos".
\chapter{Trabalhos Relacionados} \label{chap:Trabalhos}
\lettrine{P}{ara} FAZER...

\section{Revisão Sistemática da Literatura}
\lettrine{P}{ara} FAZER...

\section{Estado da arte}
\lettrine{P}{ara} FAZER...

\section{Taxonomias}
\lettrine{P}{ara} FAZER...

\section{Síntese dos trabalhos relacionados}
\lettrine{P}{ara} FAZER...

\section{Considerações finais}
\lettrine{P}{ara} FAZER...
\chapter{Proposta} \label{chap:Proposta}
\lettrine{O} uso de tecnologias na educação está cada vez maior no mundo todo, principalmente por influência da internet, é fácil conectar computadores e até mesmo de dispositivos móveis como celulares para conseguir qualquer conteúdo educativo ou informação importante. Contudo, algumas pessoas ainda encontra dificuldade na utilização destas tecnologias, do qual se destaca pessoas com necessidades visuais.

\section{Proposta}
\lettrine{A} Proposta consiste em criar um tutor que auxilie no uso da plataforma para deficientes visuais.

Para a inserção do tutor será criada uma interface totalmente voltada para deficientes visuais com os recursos de reconhecimento de voz, leitor de texto, entre outros. Dentre os diversos LMS(Learning Management Systems) existentes, foi selecionado o Moodle, por ser um dos mais utilizados segundo Coelho(2011) \citep{coelho2011acessibilidade}.


\section{Trabalhos Relacionados}
\lettrine{U}{m} trabalho que inspirou a criação de uma interface totalmente voltada para deficientes visuais e usando a base do Moodle como referência foi o trabalho do André Luiz Andrade \citep{rezende2005abaco}. O trabalho do Kraus, Helton e Anita \citep{kraus2007desenvolvimento} trouxeram a ideia da criação de um chatterbot com a técnica de RBC. O trabalho do Azeta, Ayo, Atayero and Ikhu-Omoregbe \citep{azeta2009case} trouxe a ideia de juntar a técnica de RBC para auxiliar deficientes visuais.

\section{Descrição da proposta}
\subsection{Interface}
\lettrine{P}{essoas} que não possuem necessidades visuais podem utilizar a plataforma padrão normalmente, apenas pessoas com necessidades especiais utilizarão a nova interface, assim não interferindo no uso e interação daqueles já habituados com o sistema anterior como exemplificado na imagem abaixo.

\begin{figure}[htbp!]
\centering
\includegraphics[scale=0.7]{img/uso.png}
\caption{Exemplo do uso da interface}
\label{fig:uso}
\end{figure}

A interface irá acessar os dados da base do Moodle, assim para os professores e alunos que não possuem deficiência visual não haverá diferença, excluindo a necessidade de uma nova aprendizagem para todos os envolvidos.

\subsection{Tutor}
\lettrine{N}{a} utilização do sistema, o tutor irá fazer todo o trabalho de auxílio para o deficiente visual, suas tarefas principais se resumem em auxiliar no uso da plataforma e ajustar o conteúdo para o usuário.

\begin{figure}[htbp!]
\centering
\includegraphics[scale=0.6]{img/tutor.png}
\caption{Trabalho do Tutor}
\label{fig:tutor}
\end{figure}

\subsection{Ações do Tutor}
\lettrine{O} tutor está dividido em dois agentes, o primeiro é um chatterbot que resolver as questões de uso da plataforma utilizando a técnica de raciocínio baseado em casos(RBC), o segundo agente está integrado a plataforma e conforme o usuário utiliza, novos conteúdos são sugeridos também baseado na técnica de RBC.

\begin{figure}[htbp!]
\centering
\includegraphics[scale=0.6]{img/Agente.png}
\caption{Ações do Tutor}
\label{fig:agente}
\end{figure}

\subsection{Raciocínio Baseado em Casos(RBC)}
\lettrine{A} ideia principal do RBC é resolver cada novo problema lembrando das soluções de situações anteriores similares. A definição clássica de um sistema RBC foi elaborada por Reisbeck e Schank (1989) \citep{Riesbeck}: “Um sistema RBC resolve problemas, adaptando soluções que foram utilizadas para resolver problemas anteriores”.

Um exemplo seria, uma pessoa escolhe comprar um objeto, e com base nas experiências passadas decide qual será o melhor para não passar por problemas já vividos com outros objetos.

\begin{figure}[htbp!]
\centering
\includegraphics[scale=0.6]{img/RBC.png}
\caption{Ações do Tutor}
\label{fig:Ciclo RBC}
\end{figure}

\chapter{Abordagem} \label{chap:Abordagem}
\lettrine{P}{ara} FAZER...
\chapter{Resultados Parciais} \label{chap:Resultado}
\lettrine{P}{ara} FAZER...
\chapter{Conclus�es} \label{chap-Concl}

Neste trabalho foram considerados problemas de escalonamento com penalidades de antecipa��o e atraso em ambiente monoprocessado e ambiente de m�quinas paralelas id�nticas, com tarefas independentes, de tempos de processamento arbitr�rios e pesos distintos ($P$ $||$ $\sum{\alpha_{j}E_{j}}$ $+ \sum{\beta_{j}T_{j}}$ na nota��o de $3$ campos). A fundamenta��o te�rica envolvendo Otimiza��o Combinat�ria, com �nfase em teoria de escalonamento, programa��o inteira e m�todos heur�sticos, juntamente com um detalhamento aprofundado do problema de escalonamento em enfoque no trabalho, envolvendo defini��es, nota��o cl�ssica, discuss�o sobre estrutura de solu��o com empates e simetria, formula��es matem�ticas existentes e trabalhos relacionados. Foram apresentadas quatro abordagens algor�tmicas aproximadas e uma exata para resolu��o do problema.

Dentre as estrat�gias aproximadas, foram considerados os m�todos de \textbf{busca local iterada}, algoritmo inicialmente proposto por Rodrigues et al. \cite{RosianeArthurEduardoMarcus:2008} e que foi adaptado para o problema considerado neste trabalho, o m�todo de \textbf{reconex�o de caminhos com busca local iterada}, cujo objetivo � encontrar melhores solu��es no espa�o de busca formado entre dois �timos locais e, o terceiro m�todo trata-se de um m�todo de otimiza��o global com busca m�ltipla baseada em popula��o que foi desenvolvido atrav�s do \textbf{algoritmo gen�tico com reconex�o de caminhos} onde, a cada itera��o, � gerada uma nova popula��o com solu��es cada vez melhores e diversificadas.

No quarto m�todo proposto tamb�m foi utilizada a ideia de busca m�ltipla baseada em popula��o, que foi proposto atrav�s do \textbf{algoritmo gen�tico} utilizando \textbf{busca local} com \textbf{reconex�o de caminhos}, cujo objetivo foi de explorar cada vez mais o espa�o de busca entre duas solu��es �timas locais afim de convergir rapidamente para melhores solu��es. A quinta e �ltima estrat�gia algor�tmica apesentada envolveu o ILS com busca m�ltipla baseada em um conjunto de solu��es, onde o objetivo desta estrat�gia foi de verificar se o algoritmo ILS consegue atingir melhores solu��es para inst�ncias de tamanho maior, considerando um conjunto de solu��es diversificadas a cada itera��o.

O m�todo exato considerado neste trabalho foi algoritmo \textit{branch-and-cut} do CPLEX, onde foram consideradas duas formula��es matem�ticas, uma formula��o para o ambiente monoprocessado, sem tempo ocioso entre as tarefas, proposta por Tanaka et al.~\cite{Tanaka:2009:EAS:1644424.1644462}, e a outra formula��o para o ambiente de m�quinas paralelas, formula��o cl�ssica de escalonamento que, atrav�s dos experimentos computacionais apresentados, essa formula��o n�o trata o tempo ocioso em algumas inst�ncias. A implementa��o contou com a utiliza��o da biblioteca UFFLP para C++ onde, a partir de uma inst�ncia de teste, foi poss�vel gerar o modelo matem�tico a ser executado pelo CPLEX.

Uma an�lise comparativa emp�rica mostrou que a estrat�gia baseada em popula��o do algoritmo gen�tico com busca local e reconex�o de caminhos entre dois �timos locais foi o melhor e mais promissor m�todo proposto, atingindo todos os �timos da literatura para o ambiente de monoprocessado. Os m�todos apresentados tamb�m s�o adapt�veis ao ambiente de m�quinas paralelas, onde � poss�vel atingir boas solu��es em um tempo razo�vel. Infelizmente ainda n�o existem \textit{benchmarks} na literatura para m�quinas paralelas para compara��o com os nossos resultados, mais este trabalho apresentou uma compara��o dos resultados dos m�todos aproximados e exatos a fim de se tenha uma \textbf{refer�ncia de qual seria a melhor solu��o} ou \textbf{solu��o �tima} para o problema envolvendo \textbf{ambiente multiprocessado}, espera-se tamb�m que estes resultados possam ser utilizados como \textbf{refer�ncia em outras pesquisas} para fins de compara��o com outros m�todos.

Os resultados obtidos a partir da execu��o do m�todo exato mostraram que, quando o mesmo consegue terminar, ou seja, encontrar a melhor solu��o, a execu��o do \textit{branch-and-cut} do CPLEX � mais r�pida para m�quinas paralelas, ao contr�rio dos m�todos aproximados propostos. Isso se deve ao fato de que o tempo total de processamento tende a diminuir a medida que o n�mero de m�quinas dispon�veis aumenta, pois, para os m�todos aproximados, considera-se o custo de distribuir as tarefas nas m�quinas utilizando regras de despacho toda vez que uma solu��o tiver que ser avaliada. Mas, apesar disso, os m�todos aproximados s�o extremamente promissores para o problema, apresentando bons resultados em um razo�vel espa�o de tempo, al�m do mais, o tempo de processamento dos m�todos aproximados apresentados podem ser minimizados atrav�s da combina��o de novas estrat�gias de solu��o.

\section{Confer�ncias e publica��es}

Os trabalhos apresentados/publicados est�o listados abaixo em ordem cronol�gica do mais recente ao mais antigo:

 	\begin{itemize}
 	 \item  \textbf{Em processo de publica��o para a revista \textit{Expert Systems with Applications}}.
 	 \vspace{-0.3cm}
                \begin{itemize}
                   \item \textit{Single and multi-start methods based on local search and path-relinking technique for earliness-tardiness parallel machine scheduling problems}.
                \end{itemize}
     \item  \textbf{Trabalho no GECCO 2013 - Genetic and Evolutionary Computation Conference}.
     \vspace{-0.3cm}
                \begin{itemize}
                   \item \textit{A hybrid genetic algorithm with local search approach for E/T scheduling problems on identical parallel machines}~\cite{amorim-dias-rodrigues-uchoa:2013}.
                \end{itemize}
     \item  \textbf{Trabalho no WoPI 2012 - II Workshop de Pesquisa em Inform�tica}.
     \vspace{-0.3cm}
                \begin{itemize}
                   \item \textit{Minimiza��o da Antecipa��o e Atraso em Escalonamento de Processos Produtivos Usando M�quinas Paralelas}~\cite{amorim-rodrigues3:2012}.
                \end{itemize}
     \item  \textbf{Trabalho no EURO-INFORMS MMXIII - 26th European Conference on Operational Research}.
     \vspace{-0.3cm}
                \begin{itemize}
                   \item \textit{Algorithmic strategies to single and parallel machine scheduling problems with earliness-tardiness penalties}~\cite{amorim-rodrigues2:2013}.
                \end{itemize}
     \item  \textbf{Trabalho no EURO 2012 - $25^{th}$ European Conference on Operation Research}.
     \vspace{-0.3cm}
                \begin{itemize}
                   \item \textit{MIP models and algorithms for earliness/tardiness scheduling problems on parallel machines}~\cite{amorim-dias-rodrigues2:2012}.
                \end{itemize}
     \item  \textbf{Trabalho no CLAIO/SBPO 2012 - Congreso Latino-iberoamericano de Investigaci�n Operativa / Simp�sio Brasileiro de Pesquisa Operacional}.
     \vspace{-0.3cm}
                \begin{itemize}
                   \item \textit{ILS com reconex�o de caminhos entre �timos locais para um problema cl�ssico de escalonamento com antecipa��o e atraso}~\cite{amorim-dias-rodrigues:2012}.
                \end{itemize}
 	 \item  \textbf{Participa��o na ELAVIO 2012 - XVI Escuela Latinoamericana de Verano en Investigaci�n Operativa}.
 	 \vspace{-0.3cm}
                \begin{itemize}
                   \item \textit{Problemas cl�ssicos de escalonamento sob restri��es de antecipa��o e atraso}~\cite{amorim-rodrigues:2012}.
                   \item Participa��o na din�mica de grupo.
                \end{itemize}
    \end{itemize}

Al�m dos eventos e publica��es supracitadas, foram realizadas visitas t�cnicas em institui��es parceiras do Rio de Janeiro, incluindo o Departamento de Engenharia de Produ��o da Universidade Federal Fluminense, o Programa de Engenharia de Sistemas e Computa��o da COPPE/Universidade Federal do Rio de Janeiro e a sede do Instituto Nacional de Metrologia, Qualidade e Tecnologia (Inmetro), onde houve um encontro t�cnico com exposi��es tanto de pesquisas realizadas neste trabalho quanto de membros da institui��o.


\section{Trabalhos futuros}

Para trabalhos futuros envolvendo problemas de escalonamento com penalidades de antecipa��o e atraso, considera-se importante serem  propostas formula��es matem�ticas em programa��o inteira mais robustas para tais problemas, de tal forma a serem desenvolvidos m�todos exatos mais eficientes. M�todos h�bridos, envolvendo a resolu��o de formula��es relaxadas para o problema em conjunto com m�todos heur�sticos, � um caminho interessante a seguir, no intuito de se obter solu��es �timas com tempos de execu��o mais eficientes. 

Um aprofundamento no estudo e tratamento de solu��es com empate e sim�tricas tamb�m faz-se necess�rio, de modo que procedimentos algor�tmicos que sejam capazes de detectar tais solu��es equivalentes sejam elaborados. Assim, tais solu��es equivalentes n�o precisam ser consideradas durante o processo de busca por melhores solu��es, permitindo uma redu��o significativa do tempo de processamento do algoritmo. Por outro lado, � interessante tamb�m investigar novos ambientes de processamento como m�quinas de prop�sito geral (\textit{mpm}) e diferentes tipos de restri��es, tais como: tempos de prepara��o (\textit{setups}), datas de t�rmino iguais (\textit{common due dates}) e preemp��o (\textit{pmtn}).


\bibliographystyle{alpha-ime}
\label{bib:begin}
\bibliography{references}
\label{bib:end}
\end{document}
