\chapter{Trabalhos Relacionados} \label{chap:Trabalhos}
\lettrine{O} uso de tecnologias na educação está cada vez maior no mundo todo, principalmente por influência da internet, 
é fácil conectar computadores, até mesmo de dispositivos móveis como celulares para  conseguir qualquer conteúdo educativo ou 
informação importante. Contudo, algumas pessoas ainda encontra dificuldade na utilização destas tecnologias, do qual se destaca 
pessoas com necessidades visuais.

Para esse público específico a quantidade de plataforma que está preparada para receber este tipo de usuário ainda é limitada, e no 
ramo da educação a distância o impacto é maior.
Alguns autores buscam constantemente soluções para incluir este tipo de usuário, neste trabalho algumas soluções já propostas
serão relembradas.

%%%%%%%%%%%%%%%%%%%%FAZER%%%%%%%%%%%%%%%%%%%%%%%%
%%
%% \section{Revisão Sistemática da Literatura}
%% \lettrine{P}{ara} FAZER...
%%
%% \section{Estado da arte}
%% \lettrine{P}{ara} FAZER...
%%
%% \section{Taxonomias}
%% \lettrine{P}{ara} FAZER...
%%
%% \section{Síntese dos trabalhos relacionados}
%% \lettrine{P}{ara} FAZER...
%%
%%%%%%%%%%%%%%%%%%%%%%%%%%%%%%%%%%%%%%%%%%%%%%%%%

\section{Ambientes virtuais de aprendizagem}
\lettrine{A} vasta facilidade de acesso a internet e da necessidade de ambientes educacionais que suportem 
adequadamente as atividades dos alunos e professores, a Web, se apresenta como um recurso pronto para atender 
parte das expectativas dos entusiastas da área de Ambientes virtuais de aprendizagem.

Para se criar o projeto de qualquer sistema acessível, faz-se necessária a interação de mecanismos que disponibilizem o 
conteúdo a partir da conversão de significados, o que pode favorecer a mediação de informações semânticas na utilização 
de ambientes virtuais \cite{Paper2008}.

Na intenção de facilitar para alunos e professores, diversas plataformas tem sido desenvolvida utilizando os recursos 
disponíveis na Web. No contexto das instituições de ensino, alunos e professores percebem, através da disponibilização 
de notas de aula, listas de exercícios e trabalhos na Internet, as vantagens de ter estes recursos presente no processo 
de ensino-aprendizagem, seja ele presencial ou à distância.

\subsection{Moodle}
\lettrine{M}{oodle} (Modular Object Oriented Dynamic Learning Environment) é um Ambiente Virtual de Ensino e Aprendizagem (AVEA), Na educação a distância foi onde o Moodle se tornou muito popular, sua função básica está descrita no manual \cite{Moodle} como “um software desenhado para auxiliar educadores a organizar e gerenciar com facilidade cursos online”. 

O Moodle é um software livre, sua principal função é a de ser um sistema de administração voltado à educação, especialmente dirigido à criação de fórum em ambientes direcionados à aprendizagem.

Fazem uso do Moodle. A utilização de tais ambientes é fundamental ao processo de ensino e aprendizagem dos discentes, mas sabe-se que esse espaço virtual ainda apresenta barreiras no acesso aos conteúdos e 
outras informações, principalmente para o aluno com deficiência. A organização da informação no ambiente virtual, como 
ela chega até o acadêmico e o conteúdo são alguns exemplos de que, além da estrutura do Moodle, a ação docente 
também é imprescindível neste processo \cite{FIALHO}.

Dentre os principais benefícios do uso do Moodle, podem-se destacar:
\begin{itemize}
  \item{É gratuito, isto é, qualquer instituição ou mesmo pessoa física pode efetuar seu download e instalá-lo, sem custos;}
  \item{É open source, o que permite o desenvolvimento de soluções customizadas a partir do mesmo;}
  \item{Adota o padrão Scorm, utilizado em objetos de aprendizagem e outras tecnologias para educação;}
  \item{Apresenta uma grande comunidade de usuários, o que permite a interação e troca de experiências entre os mesmos.}
\end{itemize}

Um projeto que possui bastante citações e tem foco no Moodle e acessibilidade é 
o Easy \cite{rezende2005abaco}, produto gerado a partir da tese de mestrado do professor André Luiz Andrade, no qual usa os dados do Moodle como base e cria um nova interface totalmente voltada para os deficientes visuais.

\section{Acessibilidade na educação}
\lettrine{D}{iversos} trabalhos vêm sendo realizados, em diferentes linhas de pesquisa, com o objetivo de auxiliar pessoas cegas a 
navegar na Web. Alguns destes trabalhos procuram identificar o impacto das diretrizes de acessibilidade propostas pelo 
W3C no desenvolvimento de sites acessíveis, assim como a eficácia dessas diretrizes frente aos problemas encontrados 
pelos usuários cegos \cite{4ce5a8d646ac449e98d079b8f6c3a7c5} \cite{Power2012}.

Outras pesquisas têm como objetivo identificar o comportamento dos usuários cegos ao navegar na Web. 
\cite{Vigo2013} analisaram como os usuários cegos se comportam frente a diversas situações problemáticas 
encontradas ao navegar na Web, com o objetivo de possibilitar uma melhoria nos métodos existentes de avaliação 
e modelagem Web. \cite{Voykinska2016} exploram as motivações, desafios, interações e experiências das 
pessoas cegas ao interagir com conteúdos visuais em redes sociais.

Diversos trabalhos abordam questões específicas para a acessibilidade na Web, como por exemplo, 
a acessibilidade de vídeos online com base em anotações de vídeo e enriquecimento de áudio composto por 
síntese de voz e ícones sonoros \cite{Encelle2011} e a exploração de dados georreferenciados 
por meio de coordenadas de mapas, sons não textuais e saída de voz \cite{Zhao2008}.

Por fim, algumas pesquisas adotam o conceito de que as interfaces projetadas para usuários com 
visão podem não se adequar de maneira satisfatória aos usuários cegos, mesmo que em conformidade 
com as diretrizes de acessibilidade. Assim, são encontrados trabalhos na literatura com o enfoque 
do design elaborado especificamente para as pessoas cegas. \cite{Press2008} desenvolveram uma 
interface do usuário textual (Enhanced Text user Interface - ETI) como uma alternativa 
às interfaces gráficas (GUI). Para avaliar essa nova interface, foi realizado um experimento com 39 
usuários cegos, o qual apresentou um aumento de velocidade na tarefa de busca de conteúdo. Porém, a 
ETI não apresentou melhora na tarefa de navegação, possivelmente devido a problemas relacionados à 
falta de classificação do conteúdo.

Em uma outra abordagem, denominada ABBA \cite{Fayzrakhmanov2010}, a página Web é transformada 
em um modelo semântico formal multi-axial, em que os eixos oferecem meios para serializar o 
documento de acordo com diferentes visões semânticas. Com isso, uma pessoa cega pode navegar 
pelo conteúdo e ir diretamente aos eixos desejados, sendo direcionada para as partes relevantes 
da página. Porém, ainda em desenvolvimento, o ABBA possui um conjunto limitado de eixos desenvolvidos, 
e alguns desses eixos são definidos apenas manualmente.

O conceito de uma solução técnica para filtrar informações redundantes e irrelevantes durante a navegação 
de pessoas cegas na Web é proposta por \cite{Steiner2015}. Esta solução é baseada em um 
algoritmo que analisa o conteúdo HTML das páginas Web, compara os elementos da página com os elementos das 
páginas visitadas anteriormente e com os elementos de um banco de dados compartilhado, com objetivo 
de filtrar informações que não são necessárias aos usuários do leitor de tela. 
\cite{Authors2016} apresentaram um serviço de proxy que adapta as páginas Web e as apresenta de uma 
forma usável aos usuários com deficiência visual. O proxy VIPaware apresenta um serviço que recupera as 
principais informações a partir de qualquer página da Web, analisa o HTML da página solicitada e cria 
um DOM (Document Object Model), que é tratado para eliminar elementos inacessíveis.

A W3C, por meio da WAI (Web Accessibility Initiative), trabalha em diversos padrões
e recomendações que têm por objetivo melhorar a acessibilidade de sites. No entanto,
por focar seus esforços em conformidade com suas recomendações, o W3C promove uma
visão mais técnica e mensurável da acessibilidade. Porém, nem sempre ao atingir um
bom nível de acessibilidade segundo esses padrões, um site ou aplicação possui uma boa
usabilidade \cite{Petrie2007}.

\section{Técnicas de inteligência artificial}
\lettrine{A} introdução de técnicas de inteligência artificial nestes ambientes tem diversas finalidades, 
sendo alguma delas;  acompanhamento do aluno, modelos do processo de ensino-aprendizagem melhores; maior 
possibilidade de manter o foco e concentração do aluno.

Atualmente existem diversas técnicas de inteligência artificial, e vários artigos mostram formas de utilização, 
voltada para o ambiente da educação, como MultiAgentes, RBC, entre outros.

A tecnologia de agentes também tem sido aplicada no projeto de outros tipos de sistemas educacionais, 
com alguma ênfase no uso de agentes em Ambientes Virtuais de Aprendizagem (AVA) e no suporte ao uso de OA. 
Como exemplos recentes de aplicações na área de AVA destacam-se os trabalhos \cite{Arias}  e \cite{Campana2008} que 
apresentam propostas arquiteturas multiagente para AVA. No contexto da aplicação da tecnologia de agentes para OA, 
se destaca a proposta dos objetos de aprendizagem inteligentes 
(ILO-Intelligent Learning Objects) \cite{Gomes2004} \cite{Bavaresco2009}.

A Web Semântica busca descrever o conteúdo dos recursos da Web atual, com o objetivo de dar suporte, 
tanto para agentes humanos quanto para agentes artificiais, no processamento de informações. 
Diante desta perspectiva, a comunidade de Inteligência Artificial e Educação tem demonstrado 
interesse neste ramo de pesquisa, com o ¸ intuito de evoluir os sistemas educacionais atuais,
criando ambientes que sejam adaptativos e semânticos.

A Web Semântica, oferece diversas melhorias no contexto de Sistemas Educacionais Baseados na Web, 
contribuindo para o aumento da qualidade da aprendizagem, criando assim os Ambientes Educacionais 
Baseados na Web Semântica.

A aplicação de tecnologias relacionadas a Web Semântica no projeto de ambientes e sistemas educacionais é um fenômeno atual \cite{Isotani} \cite{Proceedings2007}. Apesar disso, a tecnologia de engenharia de ontologias, 
tem se mostrado útil na concepção de vários tipos de ambientes educacionais, incluindo, entre outros, autoria 
de conteúdos \cite{Isotani} \cite{Isotani2008}, ambientes web \cite{Silva2009} \cite{Bittencourt2009} e 
modelos educacionais formais \cite{Hayashi2009}. Nesse contexto, ontologias são tipicamente empregadas para 
definir as propriedades dos elementos e entidades relativas ao sistema educacional. Há uma tendência a seguir 
a estruturação dos ITS e dividir as ontologias educacionais em três tipos \cite{Silva2009}: a) ontologias 
para o domínio de ensino, b) ontologias sobre métodos pedagógicas, e c) ontologias a respeito do modelo de aluno. 
Alternativamente, existem propostas bastante detalhadas de ontologias que integram aspectos dos três tipos citados acima, 
sendo capazes, por exemplo de modelar as várias propriedades de um processo de aprendizagem em diversos níveis de 
granularidade \cite{Proceedings2007} ou descrever processos de aprendizagem colaborativa \cite{Hayashi2009}.

\section{Considerações finais}
\lettrine{T}{odos} os trabalhos abordam formas e ferramentas diferentes para a inclusão do deficiente visual no mundo da educação. 
Todas as técnicas tem seus méritos, e no momento da criação resolveu o problema daquele momento, porém apenas incluí-los 
não auxílio na evolução dele na sociedade. Hoje temos várias formas de trabalhar e progredir nos estudos, com vários métodos 
que auxiliam o aluno no avanço da barreira do conhecimento, mas para as pessoas com necessidades especiais ainda é limitado 
os recursos.

Mesmo quando existe uma ferramenta que auxilie a usar a ferramenta de ambiente virtual, como exemplo, um leitor de 
tela ou uma ferramenta de áudio, a parte da educação, muitas vezes, é deixada de lado.