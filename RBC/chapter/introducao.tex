\chapter{Introdução} \label{chap:Introdução}
\lettrine{T}{odo} mundo tem o direito à educação, e hoje obter informação é algo tão simples, porém mesmo sendo algo simples para a maioria das pessoas, existe as pessoas com necessidades especiais que na maiorias das vezes é deixado de lado, e dentre as várias necessidades existentes, a deficiência visual é a menos tratada no mundo da informação compartilhada, pois é a mais difícil, tendo em vista que todo o sistema tem que ser de fácil acesso por um leitor de tela.

Com o passar do tempo surgiram várias formas de concluir cursos a distância, hoje é uma das principais formas de estudo, porém pessoas com deficiência visual ainda tem grande dificuldade em utilizar todos os recursos das plataformas EAD Um grande obstáculo para todos os estudante é a busca por conhecimento.

Raciocínio Baseado em Casos , ou Case Based Reasoning (CBR), é uma técnica que busca a solução para uma situação atual através da recuperação e adaptação de soluções passadas semelhantes, dentro de um mesmo domínio do problema. O sistema é capaz de localizar e encontrar partes de casos que não se adequem ao problema, criando um novo caso para uso posterior.

Moodle é uma plataforma de aprendizagem concebida para proporcionar aos educadores, administradores e alunos um único sistema robusto, seguro e integrado para criar ambientes de aprendizagem personalizados.

As pessoas com deficiência visuais que estão fora do sistema educacional e, consequentemente, sem acesso à cultura na vida adulta, podem encontrar dificuldades para conquistar a sua independência pessoal, sendo assim, pouco ou nada contribuirão ao país.

Aplicando RBC em uma plataforma de ensino online (Moodle) é possível transformar um ambiente simples em um ambiente inteligente que aprende com 1o conhecimento do aluno e auxiliar na escolha de quais níveis avançar, baseado em casos já existentes e adicionando novos casos a cada interação com o sistema.


\section{Contexto e descrição do problema}
\lettrine{U}{sar} um sistema, mesmo para uma pessoa com todos os sentidos ainda sim é algo complicado, e por tanto para alguém que possui uma necessidade especial tão difícil como a deficiência visual e uma barreira bem maior conseguir usar e usufruir de todos os recursos do sistema, tornando o uso de um ambiente virtual quase impossível de ser usado.

Hoje temos diversos recursos de leitores de tela que facilitam o uso de diversas ferramentas, mas ainda assim não é o suficiente para trazer toda a experiencia necessária para concluir com qualidade cursos a distancia, plataformas como o Moodle possui diversos plugins para auxiliar na necessidade dos alunos, oferecendo leitores de telas e dicas.

Mas mesmo com todos os recursos disponíveis, varias funcionalidades passam despercebidas.

\section{Motivação}
\lettrine{P}{ara} FAZER...

\section{Objetivos}

\subsection{Objetivo Geral}
\lettrine{P}{ropor} um tutor inteligente que use Raciocínio Baseado em Casos, para auxiliar
no uso da plataforma Moodle, específico para deficientes visuais.

\subsection{Objetivos específicos}
\lettrine{O}{s} objetivos específicos incluem:
\begin{itemize}
    \item Fazer um levantamento do que já existe de acessibilidade em Ambientes Virtuais de Aprendizagem para pessoas com deficiência visual;
    \item Efetuar reconhecimento de um padrão em uma sequência de quadros;
    \item Desenvolver técnicas para seleção do melhor caminho para aprendizagem
    de cada aluno;
\end{itemize}

\section{Justificativa}
\lettrine{P}{ara} FAZER...

\section{Questões de Pesquisa}
\lettrine{P}{ara} FAZER...

\section{Metodologia}
\lettrine{P}{ara} FAZER...

\section{Cronograma}
\lettrine{P}{ara} FAZER...

\section{Organização do documento}
\lettrine{P}{ara} FAZER...