\section{Introdução}
Todo mundo tem o direito à educação, e hoje obter informação é algo tão simples, 
porém mesmo sendo algo simples para a maioria das pessoas, existe as pessoas com 
necessidades especiais que na maiorias das vezes é deixado de lado, e dentre as 
várias necessidades existentes, a deficiência visual é a menos tratada no mundo 
da informação compartilhada, pois é a mais difícil, tendo em vista que todo o sistema 
tem que ser de fácil acesso por um leitor de tela.

Com o passar do tempo surgiram várias formas de concluir cursos a distância,
hoje é uma das principais formas de estudo, porém pessoas com deficiência visual
ainda tem grande dificuldade em utilizar todos os recursos das plataformas EAD
Um grande obstacolo para todos os estudante é a busca por conhecimento,

Raciocínio Baseado em Casos , ou Case Based Reasoning (CBR), é uma técnica
que busca a solução para uma situação atual através da recuperação e adaptação
de soluções passadas semelhantes, dentro de um mesmo domínio do problema.
O sistema é capaz de localizar e encontrar partes de casos que não se adequem
ao problema, criando um novo caso para uso posterior. \cite{Fernandes}

Moodle é uma plataforma de aprendizagem concebida para proporcionar
aos educadores, administradores e alunos um único sistema robusto, seguro e
integrado para criar ambientes de aprendizagem personalizados. \cite{Moodle}

As pessoas com deciência visuais que cam fora do sistema educacional
e, consequentemente, sem acesso à cultura na vida adulta, podem encontrar
dificuldades para conquistar a sua independência pessoal, sendo assim, pouco
ou nada contribuirão ao país.

Aplicando RBC em uma plataforma de ensino online (Moodle) é possível
transformar um ambiente simples em um ambiente inteligente que aprende com
1o conhecimento do aluno e auxiliar na escolha de quais níveis avançar, baseado
em casos já existentes e adicionando novos casos a cada interação com o sistema.