\section{Implementando UUID no PostgreSQL}
O PostgreSQL já possui um tipo de dado para UUID, então para usar como identificador basta seguir a SQL abaixo:

\begin{table}[h]
\centering
\caption{Usuario}
\vspace{0.5cm}
\begin{tabular}{r|lr}
 
Linha & Script \\ % Note a separação de col. e a quebra de linhas
\hline                               % para uma linha horizontal
1 & CREATE TABLE usuario (\\
2 & \hspace{1cm} id uuid primary key,\\
3 & \hspace{1cm} nome varchar(50), \\
4 & \hspace{1cm} idade int \\
5 & );           % não é preciso quebrar a última linha
 
\end{tabular}
\end{table}

Com este script a tabela é criada usando o UUID como chave primaria,
o PostgreSQL por padrão já tem o tipo UUID como tipo de dado, mas 
para gerar UUID por ser uma sequencia aleatoria, é necessario habilitar uma extenção,
segue abaixo o script:

\begin{table}[h]
\centering
\caption{Habilitando UUID}
\vspace{0.5cm}
\begin{tabular}{r|lr}
 
Linha & Script \\ % Note a separação de col. e a quebra de linhas
\hline                               % para uma linha horizontal
1 & CREATE EXTENSION IF NOT EXISTS "uuid-ossp";\\
2 & SELECT uuid\_generate\_v4();
 
\end{tabular}
\end{table}

Ao executar a função uuid\_generate\_v4(), será retornado um valor parecido com o valor abaixo
b1727c50-dcf9-4f65-8e9c-8be3fa68e997 esse valor é um UUID, e todas as vezes que executar essa função sera gerado um novo valor.

 
 