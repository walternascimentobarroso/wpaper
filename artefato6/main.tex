\documentclass{article}
\usepackage[utf8]{inputenc}

\title{TUTOR INTELIGENTE COM RACIOCÍNIO BASEADO EM CASOS PARA DEFICIENTES VISUAIS NA PLATAFORMA MOODLE}
\author{Walter Nascimento Barroso}
\date{Janeiro 2017}

\usepackage{natbib}
\usepackage{graphicx}
\usepackage{lettrine} 

\begin{document}

\maketitle

\section{Introdução}
\lettrine{O} uso de tecnologias na educação está cada vez maior no mundo todo, principalmente por influência da internet, é fácil conectar computadores e até mesmo de dispositivos móveis como celulares para conseguir qualquer conteúdo educativo ou informação importante. Contudo, algumas pessoas ainda encontra dificuldade na utilização destas tecnologias, do qual se destaca pessoas com necessidades visuais.

\section{Educação Inclusiva de Deficientes Visuais}
\lettrine{A} construção de práticas pedagógicas inclusivas, que promovam o acesso aos serviços e recursos pedagógicos e de acessibilidade, viabilizam a superação da discriminação e da segregação no contexto das Instituições de Ensino. De acordo com Saviani (2008) \citep{mattioli2013saviani}, a partir do processo de democratização da educação se evidencia o paradoxo inclusão/exclusão, quando os sistemas de ensino universalizam o acesso, mas continuam excluindo indivíduos e grupos considerados fora dos padrões homogeneizadores da escola. "A construção de uma sociedade de plena participação e igualdade tem como um de seus princípios a interação efetiva de todos os cidadãos. Nesta perspectiva é fundamental a construção de políticas de inclusão para o reconhecimento da diferença e para desencadear uma revolução conceitual que conceba uma sociedade em que todos devem participar, com direito de igualdade e de acordo com suas especificidades". (Conforto e Santarosa, 2002) \citep{conforto2002acessibilidade}.

\section{Ambiente Virtual de Aprendizagem}
\lettrine{O}{s} Ambiente Virtual de Aprendizagem (AVA) são espaços virtuais que permitem aos sujeitos envolvidos nos processos de ensino, aprendizagem e avaliação a busca por conhecimentos e capacitação (Tortoreli, 2012) \citep{tortoreli2012interaccao}. São projetados para facilitar o acesso a materiais de aprendizagem e a comunicação dos estudantes entre si e com os professores. Em resumo, refere-se a um espaço no qual ocorre a comunicação pedagógica em processos formativos semipresenciais ou a distância (ADELL; BELLVER; BELLVER, 2010) \citep{palacio2016analise}. Segundo Almeida (2003) \citep{de2003ambientes}, os ambientes virtuaispermitem integrar múltiplas mídias, linguagens e recursos, apresentar informações de maneira organizada e desenvolver interações entre pessoas.

Existem várias ferramentas que auxiliam no uso de ambiente virtual de aprendizagem uma que se destaca é o Moodle, Moodle é uma plataforma de aprendizagem concebida para proporcionar aos educadores, administradores e alunos um único sistema robusto, seguro e integrado para criar ambientes de aprendizagem personalizados. (Moodle, 2017) \citep{MoodleOrg}

\subsection{Moodle}
\lettrine{M}{oodle} (Modular Object Oriented Dynamic Learning Environment) é um Ambiente Virtual de Ensino e Aprendizagem (AVEA), Na educação a distância foi onde o Moodle se tornou muito popular, sua função básica está descrita no manual como “um software desenhado para auxiliar educadores a organizar e gerenciar com facilidade cursos online”. 

\section{Raciocínio Baseado em Casos}
\lettrine{A} ideia principal do RBC é resolver cada novo problema lembrando das soluções de situações anteriores similares. A definição clássica de um sistema RBC foi elaborada por Reisbeck e Schank (1989) \citep{Riesbeck}: “Um sistema RBC resolve problemas, adaptando soluções que foram utilizadas para resolver problemas anteriores”.

Um exemplo seria, uma pessoa escolhe comprar um objeto, e com base nas experiências passadas decide qual será o melhor para não passar por problemas já vividos com outros objetos.

\section{Proposta}
\lettrine{A} partir da técnica de RBC criar um tutor que auxilie os deficientes visuais no uso da plataforma e no avanço de conhecimento, no qual o tutor ira auxiliar dando dicas de conteúdos.
Como as interfaces atualmente são focadas para uma interação visual, proponho a criação de uma interface que use os dados da plataforma Moodle, fazendo com que fique transparente a interação de dados entre os usuários e totalmente focado para pessoas com deficiências visuais.

\begin{figure}[h!]
\centering
\includegraphics[scale=0.7]{exemplo.png}
\caption{Exemplo da proposta}
\label{fig:univerise}
\end{figure}

\section{Conclusão}
\lettrine{D}{ois} trabalhos são bem relevantes para esta proposta, O primeiro \citep{azeta2009case}  trabalha com pessoas com necessidades especiais e a técnica RBC e o segundo \citep{rezende2005abaco} trabalha com pessoas com necessidades especiais e a plataforma Moodle, o principal objetivo deste é trazer as melhorias que a técnica de RBC proporcionam como apresentado no primeiro trabalho para a plataforma Moodle. O Moodle por ser um ambiente bastante utilizado por diversas instituições como mostra o segundo trabalho.

\bibliographystyle{plain}
\bibliography{references}
\end{document}

